\chapter{Manuale utente}
In questo appendice viene descritta la procedura di configurazione del programma.

\section{Installazione}
Il codice sorgente e i compilati per Fedora e Windows sono ospitati su GitHub al seguente indirizzo: \verb|github.com/mikymaione/MAMEStreamingPlatform|.

\section{Configurazione}
Nella cartella \verb|./roms/| vanno inserite tutte le ROM dei giochi che si vuole rendere disponibili.\\
Nella cartella \verb|./Streaming/HTML/roms/| vanno inserite le copertine dei giochi disponibili in formato PNG, con dimensione $222 \times 315$ px.\\
Nel file \verb|./Streaming/HTML/roms/list.txt| vanno inserite le informazioni del gioco, nel seguente formato \verb|rom_name;description;other_info| ad esempio:\\ \verb|sfiii3nr1;Street Fighter III: 3rd Strike;Capcom - 1999|.

\section{Esecuzione} \label{sec:chapMU_Esecuzione}
Per avviare il MAME CGP bisogna eseguire il comando:\\
\verb|./mame64 -streamingserver -video accel -sound sdl -resolution 640x480@30|\\
Per Linux e Windows sono forniti nel progetto i due scripts \verb|run.sh| e \verb|run.bat|.