%license:BSD-3-Clause
%copyright-holders:Michele Maione
%============================================================
%
%	Piattaforma di cloud gaming per giochi arcade
%
%============================================================

% !TeX TXS-program:compile = txs:///pdflatex/[--shell-escape]

\documentclass[openany]{book}

\usepackage[T1]{fontenc}

\def\myCDL{Corso di Laurea Magistrale in Informatica}
\def\myTitle{Piattaforma di cloud gaming per giochi arcade}

\def\myName{Michele Maione}
\def\myMat{Matr. Nr. 931468}

\def\myRefereeA{Prof. Dario Maggiorini}
\def\myRefereeB{Prof. Davide Gadia}

\def\myYY{2020-2021}

%\def\thesubsection{\Roman{subsection}} %numeri romani sottosezione

% Il seguente comando introduce un elenco delle figure dopo l'indice (facoltativo)
%\figurespagetrue

% Il seguente comando introduce un elenco delle tabelle dopo l'indice (facoltativo)
%\tablespagetrue

% Package di formato
\usepackage[a4paper]{geometry}		% Formato del foglio
\usepackage[italian]{babel}			% Supporto per le lingue
\usepackage[utf8]{inputenc}			% Supporto per UTF-8
\usepackage[a-1b]{pdfx}				% File conforme allo standard PDF-A (obbligatorio per la consegna)
\usepackage{float}
\usepackage{emptypage}
\usepackage{tablefootnote}

\title{Piattaforma di cloud gaming per giochi arcade}
\author{Michele Maione}

% Package per la grafica
\usepackage{graphicx}				% Funzioni avanzate per le immagini
\usepackage{hologo}					% Bibtex logo with \hologo{BibTeX}
%\usepackage{epsfig}				% Permette immagini in EPS
%\usepackage{xcolor}				% Gestione avanzata dei colori

% Package tipografici
\usepackage{amssymb,amsmath,amsthm} % Simboli matematici
\usepackage{listings}				% Scrittura di codice

\usepackage{algorithm} 
\usepackage{algpseudocode} 

\usepackage{verbatim}
\usepackage{commath}
\usepackage{minted}

%style scritta nella bibliografia
%citestyle scritta nel testo
\usepackage[style=ext-authoryear,sorting=nty,maxbibnames=15]{biblatex}
\DeclareOuterCiteDelims{parencite}{\bibopenbracket}{\bibclosebracket}

\usepackage{csquotes}

\usepackage{caption}
\definecolor{grigio}{RGB}{100,100,100}
\captionsetup[figure]{font={footnotesize, color=grigio}}

\hypersetup{hidelinks}

\addbibresource{Bibliografia.bib}

\newcounter{savepage}
\begin{document}
% GENERAZIONE DEL FRONTESPIZIO
\begin{titlingpage}
%\begin{titlepage}
%\pagenumbering{gobble}	

\begin{center}
	{\LARGE \uppercase{Università degli Studi di Milano}}\\
	\uppercase{Facoltà di Scienze e Tecnologie}\\
	\vskip 0.4 cm
	\uppercase{Dipartimento di Informatica\\Giovanni Degli Antoni}\\
	\vskip 0.6 cm
	\centerline{\includegraphics[height=30mm]{immagini/unimi}}
	\vskip 0.4 cm
	\Large\uppercase{\myCDL}

	\vfill
	\vskip 1.5 cm

	{\Large\uppercase\expandafter{\myTitle}}
\end{center}

\vfill
\vskip 3 cm
		
\begin{tabular}{ll}
	{\large Relatore:} & \myRefereeA\\
	{\large Correlatore:} & \myRefereeB\\
\end{tabular}	

\vfill
\vskip 0.5 cm

\begin{flushright}
	\begin{tabular}{l}
		{\large Tesi di Laurea di:}\\
		{\large \myName}\\
		{\large \myMat}\\
	\end{tabular}	
\end{flushright}
	
\vfill
\vskip 3 cm

\begin{center}
	%inserire l'anno accademico
	%\sc anno accademico \myYY
	Anno Accademico \myYY
\end{center}

\end{titlingpage}
%\end{titlepage}

\frontmatter
{
\raggedleft \large %\sl 
	``Se non così, come? E se non ora, quando?''
	
	\bigskip
	
	\textemdash Primo Levi\\
}

\chapter*{Ringraziamenti}
Vorrei ringraziare il mio relatore Prof. Dario Maggiorini per il suo continuo supporto e il Prof. Davide Gadia per la sua guida utile durante i miei sforzi verso il completamento della presente tesi.
\addcontentsline{toc}{chapter}{Ringraziamenti}

%license:BSD-3-Clause
%copyright-holders:Michele Maione
%============================================================
%
%	Piattaforma di cloud gaming per giochi arcade
%
%============================================================

\chapter*{Sommario}
\addcontentsline{toc}{chapter}{Sommario}

%Il sommario deve contenere 3 o 4 frasi tratte dall’introduzione di cui la prima inquadra l’area dove si svolge il lavoro (eventualmente la seconda inquadra la sottoarea più specifica del lavoro), la seconda o la terza frase dovrebbe iniziare con le parole “Lo scopo della tesi è . . . ” e infine la terza o quarta frase riassume brevemente l’attività svolta, i risultati ottenuti ed eventuali valutazioni di questi.
Negli ultimi anni sono apparse molte piattaforme che sfruttano il paradigma del cloud computing per offrire servizi accessibili su richiesta e da remoto per archiviare file, utilizzare le suite per l'ufficio, vedere film e serie TV, ascoltare musica e a partire dal 2011 anche videogiocare.

Il cloud gaming è un servizio che unisce il cloud computing e il live streaming per rendere possibile giocare in remoto senza scaricare o installare il gioco sul device dell'utente, in pratica consente di archiviare ed eseguire i videogiochi su un server remoto e trasmettere l'output audio-video all'utente sul proprio dispositivo.

Per far conoscere alle nuove generazioni i videogiochi che hanno fatto la storia e dare la possibilità di poter giocare ancora a macchine che ormai hanno cessato di funzionare per motivi di obsolescenza, sfruttando due tecnologie entrate a far parte della quotidianità, il live streaming e il cloud computing, in questo lavoro si propone la creazione di una piattaforma di cloud gaming. La piattaforma permetterà lo streaming audio-video, direttamente e su richiesta, dei videogiochi da un server remoto ad un client (computer, console e telefono). Il gioco è archiviato, eseguito e renderizzato su un server remoto; l'input (tastiera e gamepad) viene inviato dal client al server e lì processato. Il cloud gaming permette di iniziare a giocare immediatamente poiché il gioco è già installato sul server offrendo agli utenti un rapido accesso indipendentemente dal sistema operativo e dalle capacità hardware del client utilizzato. Infine la piattaforma, indirettamente, garantisce la gestione dei diritti digitali (DRM) per gli editori. Per questo progetto verrà ampliato il software MAME (rilasciato sotto licenza GNU-GPL) che è in grado di emulare oltre 7.000 giochi arcade, in modo che possa fungere da server di cloud gaming e comunicare con un front-end HTML, rimanendo sempre indipendente dal sistema operativo, così da rendere più agevole l’installazione di uno stand per il retro-gaming.

\newpage
\setcounter{tocdepth}{2}
\tableofcontents

%license:BSD-3-Clause
%copyright-holders:Michele Maione
%============================================================
%
%	Piattaforma di cloud gaming per giochi arcade
%
%============================================================

\chapter*{Introduzione}
\addcontentsline{toc}{chapter}{Introduzione}
\markboth{}{INTRODUZIONE}

%L'introduzione deve essere atomica, quindi non deve contenere nè sottosezioni nè paragrafi nè altro. Il titolo, il sommario e l'introduzione devono sembrare delle scatole cinesi, nel senso che lette in quest'ordine devono progressivamente svelare informazioni sul contenuto per incatenare l'attenzione del lettore e indurlo a leggere l'opera fino in fondo. L'introduzione deve essere tripartita, non graficamente ma logicamente:
%Il capitolo che apre questa tesi



%Inquadramento generale
%La prima parte contiene una frase che spiega l'area generale dove si svolge il lavoro; una che spiega la sottoarea più specifica dove si svolge il lavoro e la terza, che dovrebbe cominciare con le seguenti parole "lo scopo della tesi è \dots", illustra l'obbiettivo del lavoro. Poi vi devono essere una o due frasi che contengano una breve spiegazione di cosa e come è stato fatto, delle attività sperimentali, dei risultati ottenuti con una valutazione e degli sviluppi futuri. La prima parte deve essere circa una facciata e mezza o due

Nel 1972 la società Atari pubblicava il primo videogioco della storia, Pong, vendendo 19.000 cabinati e presto molte altre società seguirono l'esempio. Alla fine del decennio iniziò l'epoca d'oro dei videogiochi arcade e la nascita delle console \parencite{High_Score}. I videogiochi uniscono narrativa, animazione e musica all'interattività, ed è grazie a quest'ultima che riescono ad esercitare un potenziale d'immersione e attrazione che gli altri media non hanno, tanto da diventare un fenomeno culturale di massa con centinaia di milioni di persone che giocano regolarmente ogni giorno, rendendoli attori dominanti nel settore dell'intrattenimento. L'importanza economica dei videogiochi arcade, negli ultimi vent'anni, è notevolmente diminuita a favore dei videogiochi per personal computer, console e più recentemente per mobile.

Il cloud computing è un paradigma a cui siamo ormai abituati e ci risulterebbe difficile abbandonare servizi come Dropbox, Office 365, Spotify e Netflix per tornare alle loro versioni "precedenti": i rullini fotografici, i documenti aziendali negli archivi, i CD audio ed i DVD a noleggio. Dalla nascita dei primi servizi di cloud computing nel 2006 alcuni oggetti sono stati sostituiti con la loro controparte informatica portando al fallimento di aziende storiche come Blockbuster (nel 2013), Kodak (nel 2012) e Borders\footnote{Borders Group era un rivenditore americano di libri e musica.} (nel 2011) \parencite{I_4_fallimenti_più_clamorosi_del_decennio}. Il cloud computing consiste nella distribuzione on-demand delle risorse IT tramite internet su tre livelli di servizio che sono: l'accesso all'infrastruttura hardware tramite API, la piattaforma software (inclusa di sistemi di sviluppo) e le applicazioni. Con il cloud computing la potenza della macchina, sia essa fisica o virtuale, aumenta automaticamente all'esigenza permettendo di gestire i picchi di utilizzo; svincola gli utenti dal dover acquistare, manutenere e gestire fisicamente le infrastrutture IT; fornisce l'accesso alle risorse informatiche da qualsiasi device, da qualsiasi luogo e in modo collaborativo.

Unendo il paradigma del cloud computing con lo streaming nasce un nuovo tipo di servizio dedicato ai videogiochi: il cloud gaming. Questo servizio rende possibile videogiocare in remoto senza scaricare o installare il gioco sul device dell'utente. Con il cloud gaming i videogiochi sono archiviati ed eseguiti su un server remoto e l'output audio-video trasmesso al dispositivo dell'utente, permettendo di iniziare a giocare immediatamente indipendentemente dal sistema operativo e dalle capacità hardware del dispositivo utilizzato. Infine, indirettamente, viene garantita la gestione dei diritti digitali (DRM) per gli editori.

Lo scopo di questa tesi è creare una piattaforma di cloud gaming per far conoscere alle nuove generazioni i videogiochi che hanno fatto la storia e dare la possibilità di poter giocare ancora a macchine che  ormai hanno cessato di funzionare per motivi di obsolescenza. Per far ciò verrà ampliato il software MAME (rilasciato sotto licenza GNU-GPL) che è in grado di emulare oltre 7.000 giochi arcade, di modo che possa fungere da server di cloud gaming e comunicare con un front-end HTML, rimanendo sempre indipendente dal sistema operativo, rendendo più agevole l’installazione di uno stand per il retro-gaming.

Per ampliare il progetto MAME ho modificato le funzionalità di rendering video e missaggio audio per convogliare il loro output, che viene codificato in MPEG-TS, ad un modulo in grado di eseguire lo streaming ad una pagina HTML tramite il protocollo WebSocket. Ho creato un modulo JavaScript per la pagina web per gestire l'input utente e decodificare il filmato MPEG-TS.

La piattaforma utilizza un bit-rate tra 0.5 Mbps e 2.2 Mbps ed offre una risoluzione di 480p; per valutarne la latenza è stata testata su rete locale e su rete internet; mentre la qualità audio-video è stata classificata tramite "peek signal-to-noise ratio" e "structural similarity index method".




%Breve descrizione del lavoro
%La seconda parte deve essere una esplosione della prima e deve quindi mostrare in maniera più esplicita l'area dove si svolge il lavoro, le fonti bibliografiche più importanti su cui si fonda il lavoro in maniera sintetica (una pagina) evidenziando i lavori in letteratura che presentano attinenza con il lavoro affrontato in modo da mostrare da dove e perchè è sorta la tematica di studio. Poi si mostrano esplicitamente le realizzazioni, le direttive future di ricerca, quali sono i problemi aperti e quali quelli affrontati e si ripete lo scopo della tesi. Questa parte deve essere piena (ma non grondante come la sezione due) di citazioni bibliografiche e deve essere lunga circa 4 facciate.

%intro cloud gaming
Cloud for Gaming si riferisce all'uso delle tecnologie di cloud computing per costruire infrastrutture di gioco su larga scala, con l'obiettivo di migliorare la scalabilità e la reattività, migliorare l'esperienza dell'utente e abilitare nuovi modelli di business \parencite{Cloud_for_Gaming}.

L'uso dei giochi digitali nell'educazione è oggetto di ricerca da molti anni e la loro utilità è stata confermata da molti studi e progetti di ricerca. Test standardizzati, come il test PISA, mostrano che gli intervistati hanno ottenuto risultati migliori in lettura, matematica e fisica se hanno utilizzato maggiormente il computer per attività legate al gioco. È stato dimostrato che l'applicazione dei videogiochi nell'istruzione aumenta la motivazione degli studenti, migliora diversi tipi di abilità chiave: abilità sociali e intellettuali, riflessi e concentrazione. Tuttavia, ci sono diverse sfide associate all'applicazione dei videogiochi nelle scuole e possono essere classificate come tecniche (limitazioni della rete e dei dispositivi finali), competenza (conoscenza degli insegnanti nell'area), qualitativa (mancanza di giochi educativi di alta qualità) e finanziario (costo elevato per l'acquisto di giochi e attrezzature). La nuova architettura per la fornitura di contenuti di gioco comunemente denominati "cloud gaming" ha il potenziale per risolvere la maggior parte delle attuali sfide dell'utilizzo dei giochi nell'istruzione. Una piattaforma di cloud gaming ben progettata consentirebbe un utilizzo semplice e senza interruzioni sia per studenti che per insegnanti. Pur risolvendo la maggior parte dei problemi attuali, il cloud gaming introduce una serie di nuove sfide di ricerca che saranno discusse in questa sezione \parencite{Use_of_Cloud_Gaming_in_Education}.


%concorrenti
La presente tesi introduce un sistema di Games on Demand che non richiede il possesso di attrezzature speciali da parte degli abbonati. Ogni gioco viene eseguito su una macchina situata nei locali del venditore e la sua visualizzazione viene trasmessa all'utente in tempo reale, mentre la reazione dell'utente viene ritrasmessa e inserita nel gioco. Poiché nessun dato viene installato o salvato sul lato dell'abbonato, il sistema salva l'abbonato dall'acquisto e dall'aggiornamento frequente di una macchina potente e garantisce che anche gli utenti con poca o nessuna conoscenza del computer possano iscriversi e utilizzare questo servizio. Il sistema supporta i giochi realizzati con la tecnologia DirectX e non richiede alcuna modifica dei giochi originali. Pertanto, qualsiasi gioco può essere utilizzato nel sistema nel suo primo giorno di rilascio. I frame del gioco vengono catturati tramite una tecnica di process hooking e codificati in tempo reale nel formato MPEG2. I frame codificati vengono incapsulati in un Transport Stream e trasmessi in streaming al client dell'utente tramite il protocollo UDP. Le azioni dell'utente vengono acquisite come input non elaborati e ritrasmesse al server di gioco tramite UDP, dove vengono inserite nelle code degli eventi del mouse o della tastiera e trasmesse automaticamente al processo di gioco dal sistema operativo. Questa tecnica consente la continua falla dei dati e rende il controllo remoto di un gioco una realtà \parencite{ARealTimeStreamingGamesonDemandSystem}.

Presentiamo il primo sistema di cloud gaming open source, chiamato GamingAnywhere. Oltre alla sua apertura, abbiamo progettato GamingAnywhere per un'elevata estensibilità, portabilità e riconfigurabilità. L'abbiamo implementato su Windows, Linux, OS X e Android. Abbiamo condotto ampi esperimenti per valutarne le prestazioni. I nostri risultati sperimentali indicano che GamingAnywhere è efficiente, scalabile, adattabile alle condizioni di rete e raggiunge un'elevata reattività e qualità di streaming. Gaming- Anywhere può essere impiegato da ricercatori, sviluppatori di giochi, fornitori di servizi e utenti finali per la creazione di banchi di prova per il cloud gaming, che riteniamo stimoleranno ulteriori ricerche sulle innovazioni per i sistemi e le applicazioni di cloud gaming \parencite{GamingAnywhere}.



% non gaming ma tecnologie
In questo lavoro di tesi si è voluto progettare un sistema di monitoraggio per la gestione delle informazioni ottenute dalla scena del campo di emergenza al fine di aiutare gli operatori e i decisori ad avere una visione più completa del campo di emergenza così da agire correttamente e non solo. velocemente.
Sono state valutate varie soluzioni per ogni elemento del sistema di monitoraggio, inclusi Media Streaming Server, Media Streaming Protocol e codec video e per ciascuna parte viene selezionato lo strumento migliore. Inoltre, una soluzione di rilevamento degli oggetti viene implementata sopra gli elementi video per aiutare il supervisore a identificare le informazioni richieste in modo più rapido e accurato \parencite{CombinedICTTechnologiesforSupervisionofComplexOperationsinResilientCommunities}.

Questo lavoro introduce un nuovo modello per i sistemi di cloud gaming volto a ottimizzare la tempestività dei frame video sulla base di una metrica dell'età delle informazioni (AoI). I client mobili inviano azioni tramite una rete di accesso a un server di gioco. Il server di gioco genera fotogrammi video a una frequenza fotogrammi costante. Sul dispositivo mobile, la visualizzazione di questi frame rappresenta gli aggiornamenti dello stato del gioco. Sviluppiamo un modello Markov per caratterizzare il processo di consegna dei frame nei sistemi di cloud gaming edge a bassa latenza. Sulla base di questo modello, deriviamo una semplice formula per l'età media dello stato di un sistema di gioco mobile a bassa latenza strettamente sincronizzato in cui il periodo di inter-frame contribuisce in modo significativo alla latenza del sistema. Convalidiamo il modello mediante simulazione ns-3 di un sistema di cloud gaming edge a bassa latenza. I nostri scenari di valutazione includevano giochi per giocatore singolo e giochi multi-giocatore in cui l'elaborazione del gioco era condotta da una combinazione di un server di gioco centralizzato e renderer edge cloud \parencite{Timely_cloud_gaming}.


%analisi Cloud gaming
Questo articolo propone un'analisi delle caratteristiche di rete del Remote Play per PlayStation 4 e PlayStation Vita, fornendo alcuni approfondimenti sulle dinamiche di rete di questo sistema in diversi tipi di gioco. Le misurazioni coinvolgono videogiochi di diversi generi, analizzando i bitrate, la dimensione del pacchetto, catturando le dinamiche dell'intera sessione di streaming. I risultati indicano che il servizio genera una grande quantità di pacchetti grandi a valle e pacchetti sempre più piccoli a monte. I dati raccolti possono aiutare a capire se questo tipo di servizio potrebbe essere offerto attraverso le normali connessioni Internet odierne \parencite{Network_Analysis_of_the_Sony_Remote_Play_System}.

In questo lavoro studiamo i servizi di cloud gaming dal punto di vista della rete. Raccogliamo più di 200 tracce di pacchetti in diverse impostazioni dell'applicazione e condizioni di rete da una rete a banda larga fino a condizioni di rete mobile scadenti, per 3 servizi di cloud gaming, vale a dire Stadia di Google, GeForce Now di NVIDIA e PS Now di Sony. Analizziamo i protocolli impiegati e il carico di lavoro che impongono alla rete. Troviamo che GeForce Now e Stadia utilizzano il protocollo RTP per lo streaming dei contenuti multimediali, con quest'ultimo che si affida alle API WebRTC standard. Risulterebbero affamati di larghezza di banda e consumano fino a 45 Mbit / s, a seconda della rete e della qualità video. PS Now invece utilizza solo protocolli non documentati e non supera mai i 13 Mbit / s. Le reti mobili 4G possono spesso sostenere questi carichi, mentre le connessioni 3G tradizionali hanno difficoltà. I sistemi reagiscono rapidamente alle condizioni di rete deteriorate e le perdite di pacchetti fino al 5\% non causano una riduzione della risoluzione \parencite{A_Network_Analysis_on_Cloud_Gaming_Stadia_GeForce_Now_and_PSNow}.

In questo articolo, conduciamo un'analisi sistematica delle piattaforme di gioco cloud all'avanguardia e mettiamo in evidenza l'unicità del loro design del framework.Misuriamo anche le loro prestazioni nel mondo reale con diversi tipi di giochi, sia per la latenza di interazione che per la qualità dello streaming, rivelando sfide critiche per l'implementazione diffusa del Cloud Gaming \parencite{Cloud_Gaming_Architecture_and_Performance}.


%analisi rete
Attualmente, i sistemi e le tecnologie di informazione e comunicazione si stanno sviluppando ad un ritmo molto elevato e stanno diventando ampiamente disponibili. A questo punto, ogni anno compaiono sul mercato sempre più programmi che consentono la comunicazione tra persone a grande distanza con l'uso delle tecnologie di rete e di Internet. Molti programmi già noti forniscono rapporti per i loro utenti con l'uso di comunicazioni affidabili gratuitamente o quasi gratuitamente; ad esempio: Skype, Viber, NetMeeting, Net Speakerphone, Team Speak e Discord. Tutti questi programmi e tecnologie sono chiamati IPtelephony; include una varietà di tecnologie che forniscono il trasferimento di dati multimediali (voce, video e vari contenuti multimediali) su reti di computer e Internet. Si basano su vari protocolli di streaming in tempo reale. Molte aziende e utenti in tutto il mondo utilizzano i servizi di telefonia IP per supportare la comunicazione remota e la comunicazione video tra le persone in tempo reale. A questo proposito, il lavoro considera le caratteristiche del funzionamento di vari protocolli che vengono utilizzati per trasmettere informazioni visive tra utenti della rete e utenti di vari programmi e sottoreti. La maggior parte di loro utilizza gli stessi protocolli di rete, ma ha modi diversi di affrontare i problemi di rete, come capacità di carico insufficiente o perdite di rete, che in un modo o nell'altro influenzano la qualità della comunicazione in tempo reale e, di conseguenza, la qualità delle informazioni visive visualizzate. In particolare, le caratteristiche di diversi protocolli di trasferimento in tempo reale di rete multimediale più diffusi (dati video con audio - videoconferenza) vengono studiate sotto determinate limitazioni e perdite. Inoltre, vengono forniti i risultati degli esperimenti, le stime ottenute dell'efficacia dei protocolli e le peculiarità nel funzionamento dei vari protocolli nelle stesse condizioni di test \parencite{Network_technology_for_transmission_of_visual_information}.

Questo documento analizza la latenza in una rete di accesso ai pacchetti ad alta velocità dotata di risorse adeguate per il traffico da macchina a macchina e di gioco online emulato. Il protocollo TCP (Transmission Control Protocol) e UDP (User Datagram Protocol) vengono utilizzati per trasmettere i dati dell'applicazione e vengono confrontati i risultati delle misurazioni dal punto di vista dell'utente finale. Nel caso TCP, le tracce del traffico sono state registrate anche in due punti della rete principale (interfaccia Gn e firewall) e il ritardo complessivo è sezionato a porzioni appartenenti a parti diverse della rete (accesso, core, backbone). La diversità dei modelli di traffico utilizzati ha permesso di trarre conclusioni riguardanti la selettività di alcune parti della rete verso diversi modelli di traffico, indicando la direzione della ricerca futura sulla riduzione della latenza per domini applicativi emergenti nelle reti legacy \parencite{Latency_analysis_for_M2M}.



%Struttura della tesi
%La terza parte contiene la descrizione della struttura della tesi ed è organizzata nel modo seguente.
%La tesi è strutturata nel modo seguente:
%Nella sezione due si mostra \dots
%Nella sez. tre si illustra \dots
%Nella sez. quattro si descrive \dots
%Nelle conclusioni si riassumono gli scopi, le valutazioni di questi e le prospettive future \dots
%Nell'appendice A si riporta \dots (Dopo ogni sezione o appendice ci vuole un punto).
%I titoli delle sezioni da 2 a M-1 sono indicativi, ma bisogna cercare di mantenere un significato equipollente nel caso si vogliano cambiare. Queste sezioni possono contenere eventuali sottosezioni.

La tesi è strutturata nel modo seguente:
\begin{itemize}
	\item Il capitolo 1 fornisce un'introduzione sulla nascita dei videogiochi e dei ricavi globali dell'industria videoludica, dà una definizione di cloud computing e di cloud gaming, fà una panoramica delle piattaforme di gioco che si sono susseguite nel tempo e delle proiezioni di mercato del settore del cloud gaming.
	\item Nel capitolo 2 verrà descritto il sistema proposto, il MAME e le sue funzioni di: rendering, missaggio audio e gestione dell'input utente.
	\item Nel capitolo 3 verranno descritte le cinque fasi aggiuntive del cloud gaming e la loro implementazione in C++ come nuovi moduli del MAME: la cattura audio-video, la codifica, la trasmissione, la decodifica e la gestione dell'input utente.
	\item Il capitolo 4 analizza le prestazioni del progetto relativamente ai tre difetti intrinseci del cloud gaming: riduzione della qualità audio-video, bit-rate richiesto e il problema della latenza.
	\item Nelle conclusioni si riassumono gli scopi, le valutazioni di questi e le prospettive future.
	\item Nell'appendice A si riporta la documentazione del progetto logico. L'appendice B contiene la documentazione della programmazione. Il listato con l’autodocumentazione relativa è riportato nell'appendice C. Il manuale utente lo si trova nell'appendice D. Nell'appendice E c'è un esempio di impiego. Infine nell'appendice F si trova il datasheet.
\end{itemize}

\newpage
\setcounter{savepage}{\arabic{page}}

\mainmatter
%license:BSD-3-Clause
%copyright-holders:Michele Maione
%============================================================
%
%	Piattaforma di cloud gaming per giochi arcade
%
%============================================================

\chapter{Il cloud gaming}
Il capitolo che apre questa tesi fa un'introduzione sulla nascita dei videogiochi, i ricavi globali dell'industria videoludica, il cloud gaming e i servizi presenti sul mercato.




\section{La nascita dei videogiochi}
Nel 1952 nei laboratori dell'Università di Cambridge, come esempio a corredo di una tesi di dottorato sull'interazione uomo-macchina, fu creato OXO, la trasposizione del tris come gioco per computer. OXO è considerato tecnicamente il primo videogioco. Nel 1958 un professore di fisica del Brookhaven National Laboratory creò un gioco, Tennis for Two, che aveva il compito di simulare le leggi fisiche relative ad una partita di tennis, lo strumento utilizzato era un oscilloscopio.

Nel 1961, sei giovani scienziati del Massachusetts Institute of Technology su un PDP-1\footnote{PDP-1: Programmed Data Processor-1, era un computer della Digital Equipment Corporation del 1959.} crearono il primo videogioco a scopo di intrattenimento: Spacewar!.

Due mesi dopo due ingegneri elettrici, N. Bushnell e T. Dabney, terminarono la loro versione di Spacewar! su larga scala (1.500 copie), ma il gioco non ebbe un grande successo a causa dell'elevata difficoltà. Bushnell, dopo l'esperimento non particolarmente riuscito, decise però di insistere nel settore dando così vita alla società Atari. Il primo gioco arcade di Atari fu il primo grande successo del settore: Pong. Pubblicato alla fine del 1972, è un gioco che riproduce approssimativamente la meccanica del ping pong. Atari vendette 19.000 cabinati di Pong e presto molte altre società seguirono l'esempio. Alla fine del decennio iniziò l'epoca d'oro dei videogiochi arcade e la nascita delle console (console che hanno fatto la storia in Fig. \ref{fig:consoles_history}).

\begin{figure}[H]
	\includegraphics[width=\linewidth]{immagini/consoles_history}
	\caption{Console iconiche, fino alla generazione otto}
	\label{fig:consoles_history}
\end{figure}

I videogiochi sono un mezzo di intrattenimento unico che combina le diverse forme d'arte, quali musica, narrativa e animazione, all'interattività. Ed è proprio questa caratteristica, l'interattività, che permette loro di esercitare un potenziale d'immersione e attrazione che altri media non hanno. Sono ormai diventati un fenomeno culturale di massa con centinaia di milioni di persone che giocano regolarmente ogni giorno, il che li rende un attore dominante nel settore dell'intrattenimento, settore in continua crescita che non ha mai subito interruzioni nel corso degli anni come mostrato in Fig. \ref{fig:valore_commerciale_giochi_globale}. Negli ultimi vent'anni l'importanza economica dei videogiochi arcade è notevolmente diminuita\footnote{Giappone, Cina e Corea mantengono una forte industria arcade ai giorni nostri.} (in viola nella figura) a favore dei videogiochi per personal computer, console e più recentemente per mobile.

\begin{figure}[H]
	\includegraphics[width=\linewidth]{immagini/valore_commerciale_giochi_globale.png}
	\caption{Ricavi globali dell'industria dei videogiochi dal 1971 al 2018 (non adeguati all'inflazione). Fonte: wikipedia.org}
	\label{fig:valore_commerciale_giochi_globale}
\end{figure}




\section{Cloud gaming}
Il cloud gaming è un servizio che unisce il cloud computing e il live streaming per rendere possibile videogiocare in remoto senza scaricare o installare il gioco sul device dell'utente, in pratica i videogiochi vengono archiviati ed eseguiti su un server remoto e l'output audio-video viene trasmesso in streaming (come un filmato) ad un client sul dispositivo dell'utente. Il client gestisce l'input del giocatore che viene inviato al server ed eseguito nel gioco. Come mostrato in Fig. \ref{fig:cloud_gaming_general_scheme} le fasi di elaborazione input, esecuzione del gioco, rendering e missaggio audio, che solitamente vengono eseguite sul dispositivo del giocatore, nel paradigma del cloud gaming vengono eseguite sul server. Sempre sul server c'è una fase di codifica sotto forma di filmato dell'output audio-video del gioco ed invio tramite rete. Sul dispositivo dell'utente ci sono solo due fasi: cattura e trasmissione al server dell'input utente e ricezione e decodifica del filmato dal server. In questo modo il dispositivo dell'utente è indipendente sia dall'hardware che dal sistema operativo a patto che il client possa essere eseguito sul dispositivo dell'utente.

\begin{figure}[H]
	\includegraphics[width=\linewidth]{immagini/cloud_gaming_general_scheme}
	\caption{Architettura del cloud gaming}
	\label{fig:cloud_gaming_general_scheme}
\end{figure}

Secondo una ricerca di Newzoo sull'industria dei videogiochi, come mostrato in Fig. \ref{fig:Newzoo_Cloud_Gaming_Revenues}, nel 2020 il mercato del cloud gaming ha generato quasi 584,7 milioni di USD di entrate, di cui il 39\% e il 29\% in Nord America e in Europa, e si prevede una crescita fino a 4,8 miliardi di USD entro il 2023, se non maggiore. Per questo sono entrate nel mercato del cloud gaming anche aziende che non sono editori o produttori di videogiochi, come Google e Amazon, come vedremo nel paragrafo \ref{StoriaDelCloudGaming}. Queste previsioni sono specchio sia di connessioni di rete sempre più veloci sia nelle case degli utenti sia in mobilità, sia del paradigma del live streaming a cui oggi (sopratutto le nuove generazioni) siamo abituati, di poter accedere instantaneamente a canzoni, video clip, serie tv, film e perché no? videogiochi! Dal punto di vista del giocatore il cloud gaming offre molti vantaggi tra cui: rendere il gioco facilmente accessibile senza la necessità di scaricarlo e installarlo localmente; compatibilità con computer, smartphone e anche con smart TV (se utilizzato con un gamepad WiFi) senza dover badare ai requisiti hardware; diverse modalità di pagamento tra cui l'acquisto di un gioco su richiesta e l'abbonamento mensile/annuale per l'utilizzo di tutta (o una parte) della libreria videoludica; funzionalità aggiuntive per sfruttare al meglio questo modello, come lo streaming della sessione di gioco, cosa che siamo già abituati a vedere, ma con l'aggiunta di dare la possibilità ad uno spettatore di entrare a far parte della propria partita, funzionalità avanzate per il multiplayer come la condivisione della visuale di gioco e dei salvataggi di gioco, ecc\dots. Ma i vantaggi ci sono anche per gli sviluppatori perché il cloud gaming riduce i costi di produzione limitando lo sviluppo e il testing ad una sola piattaforma ed inoltre risolve definitivamente un problema che esiste dai tempi delle audiocassette e dei floppy disk, la pirateria. Non ultimi i fornitori di servizi che si ritrovano con un nuovo modello di business disponibile su contenuti già esistenti. Tuttavia bisogna dire che ci sono anche degli svantaggi, di cui parleremo nel capitolo \ref{Capitolo4}: perdita della qualità audio-video a causa della compressione, larghezza di banda richiesta all'utente non soddisfabile e l'ineliminabile latenza.

\begin{figure}[H]
	\includegraphics[width=\linewidth]{immagini/Newzoo_Cloud_Gaming_Revenues}
	\caption{Previsioni per il mercato globale del cloud gaming (in dollari americani). Fonte: newzoo.com/global-cloud-gaming-report}
	\label{fig:Newzoo_Cloud_Gaming_Revenues}
\end{figure}



\subsection{Storia del cloud gaming} \label{StoriaDelCloudGaming}
Una delle prime piattaforme di cloud gaming è stata OnLive di OL2, presentato alla GDC\footnote{GDC: Game Developers Conference, una conferenza annuale per gli sviluppatori di videogiochi.} 2009 e poi lanciato sul mercato a giugno 2010 negli Stati Uniti e a settembre 2011 nel Regno Unito. I giocatori, previo pagamento di un abbonamento mensile di 15\$, potevano acquistare o noleggiare giochi sulla piattaforma oppure utilizzare quelli precedentemente acquistati su Steam\footnote{Steam è un servizio di distribuzione digitale di videogiochi della società Valve.}. Il servizio era ospitato su 5 data centers situati sul suolo americano che servivano gli utenti per vicinanza geografica; la qualità video offerta era SD\footnote{Standard definition (SD) include i formati video con rapporto 4:3 e 16:9 con 480 linee di risoluzione (raramente 576).} e HD\footnote{High definition (HD) è un formato video 16:9 con 720 linee di risoluzione.} con una larghezza di banda richiesta di 5 Mbps. Era disponibile, oltre ad un client per Windows, macOS ed Android, anche una micro console da collegare alla TV. Il servizio fu chiuso ad aprile 2015 dopo essere stato acquistato da Sony\cite{Cloud_gaming_history}.

Nel febbraio 2012 Gaikai ha inaugurato il suo omonimo servizio di cloud gaming disponibile in 12 nazioni e distribuito su 24 data centers. La società si è concentrata principalmente sull'utilizzo del cloud gaming come forma di pubblicità online per i videogiochi, dove gli utenti avrebbero avuto la possibilità di accedere alle demo dei videogiochi sponsorizzati. La piattaforma era accessibile tramite browser (utilizzando il plugin di streaming disponibile in Adobe Flash, Java o Google Native Client), la qualità video HD richiedeva una larghezza di banda minima di 5 Mbps. La società fu acquisita da Sony a luglio 2012 per integrare la tecnologia di streaming di Gaikai nella piattaforma di cloud gaming PlayStation Now di Sony\cite{Gaikai_Beta}.

PlayStation Now è un servizio di cloud gaming basato sulla tecnologia cloud di Gaikai. È stato presentato durante il CES\footnote{CES: Consumer Electronics Show, un evento annuale che ospita presentazioni di nuovi prodotti e tecnologie nel settore dell'elettronica di consumo.} 2014. È disponibile da gennaio 2015 in Nord America, da settembre in Giappone e Regno Unito ed ha iniziato ad operare in Europa gradualmente da agosto 2017 a marzo 2019. La piattaforma consente all'utente di giocare ai titoli PlayStation (attualmente dal catalogo giochi della PS2, PS3 e PS4) su PS4, PS5 e Windows\cite{PlayStation_Now}.

La piattaforma Utomik è stata lanciata in beta a giugno 2015 e commercialmente nel maggio 2018 e da allora è in servizio. I giochi per essere riprodotti in un browser richiedono del plug-in Utomik Player. La piattaforma offre un SDK, plug-in e servizi online per creare, avviare, mantenere e monitorare i giochi pubblicati\cite{Utomik}.

GeForce Now è il servizio di cloud gaming di Nvidia lanciato in beta a gennaio 2017 e ufficialmente a febbraio 2020. GeForce Now consente agli utenti di accedere da remoto (tramite streaming) a un computer virtuale, dove possono installare giochi acquistati su Steam, Ubisoft Connect\footnote{Ubisoft Connect è un servizio di distribuzione digitale della società Ubisoft.} o Epic Games Store\footnote{Epic Games Store è un negozio di videogiochi digitali gestito da Epic Games.}. Il servizio può essere utilizzato su Windows, macOS, iOS, Android o Nvidia Shield TV\footnote{Nvidia Shield TV è un lettore multimediale digitale basato su Android.}\cite{GeForce_Now}.

A maggio 2018 Electronic Arts ha svelato la sua piattaforma di cloud gaming chiamata Project Atlas, che mira a rendere disponibili numerosi titoli e a fornire un'esperienza di gioco mai provata prima grazie al supporto dell'intelligenza artificiale. La piattaforma mira ad offrire un'esperienza di gioco composta da universi realmente viventi, che cambiano con il passare del tempo, con l'interazione con altri giocatori e sotto l'influenza del mondo esterno. In questi processi, il supporto dell'intelligenza artificiale e l'apprendimento delle abitudini e delle preferenze dei giocatori giocherebbero un ruolo fondamentale. Offre anche un client di gioco dinamico, che consente agli utenti di riprodurre in streaming un titolo mentre attendono il completamento del download sul proprio dispositivo\cite{Project_Atlas}.

RemoteMyApp della Vortex è un servizio di cloud gaming lanciato a novembre 2018, è disponibile per Android, Windows e macOS e offre tre piani mensili (12\$, 23\$ e 33\$) che consentono all'utente di giocare per un massimo di 140 ore al mese ad un catalogo di 170 giochi. Sfortunatamente, alcuni giochi possono essere riprodotti solo acquistando la licenza del gioco\cite{RemoteMyApp_Vortex}.

Microsoft ha anticipato Xbox Cloud Gaming all'E3\footnote{E3: Electronic Entertainment Expo, un evento commerciale per l'industria dei videogiochi.} 2018. La piattaforma è disponibile per gli abbonati a Xbox Game Pass Ultimate da settembre 2020 ed offre sia la libreria esistente di giochi per Xbox che per Xbox Series X. Il servizio è progettato per funzionare con gli smartphone (attualmente solo Android), con controlli touchscreen o usando un controller Xbox tramite Bluetooth\cite{Xbox_Game_Pass_cloud_gaming}.

Google Stadia è una piattaforma di cloud gaming rilasciata a novembre 2019, ma è disponibile solo in Europa e negli Stati Uniti. Sulla piattaforma l'utente può acquistare giochi o iscriversi al servizio per accedere al catalogo giochi. Google ha creato il controller Stadia che si connette tramite WiFi direttamente al servizio e rende possibile giocare su una TV (installando l'app o utilizzando un Chromecast\footnote{Google Chromecast è un lettore multimediale digitale per contenuti audiovisivi in streaming su Internet}). Invece sul computer tramite il browser Chrome si può giocare con mouse e tastiera o controller. Per quanto riguarda il mobile è disponibile un'app che supporta i controlli touch screen e i gamepad Bluetooth. La piattaforma offre alcune funzionalità interessanti come: live streaming su YouTube del proprio gameplay; Crowd Play che consente agli spettatori di unirsi ai giochi multiplayer che stanno guardando; Stream Connect che consente all'utente di condividere la schermata di gioco con altri giocatori nello stesso gioco; "Condivisione dello stato" che consente ai giocatori di condividere il proprio stato di salvataggio\cite{Google_Stadia}.

Amazon Luna è stata annunciata a settembre 2020, con "accesso anticipato" disponibile per gli abbonati su invito a partire da ottobre 2020. Il catalogo giochi proposto consta di 100 giochi. La piattaforma è ovviamente ospitata su AWS\footnote{AWS: Amazon Web Services è la piattaforma di cloug computing di Amazon.}. Il servizio offre l'integrazione con Twitch e una partnership con Ubisoft che da l'accesso ai titoli al momento del rilascio\cite{Amazon_Luna}.

La società Playkey ha realizzato una piattaforma di cloud gaming distribuita, in alpha testing durante il 2021. Il sistema distribuito è formato da un server centrale che gestisce l'infrastruttura e dai computer dei cosidetti "minatori", coloro che mettono a disposizione il proprio computer come unità di calcolo del sistema distribuito. In pratica i giocatori non si connettono ad un server per ricevere lo streaming del gioco, ma al computer di un minatore, su cui viene eseguito il gioco, la codifica e lo stream. I minatori guadagnano 10\$ al giorno mentre l'utente paga 23\$ mensilmente per il piano illimitato. I giocatori possono giocare solo i titoli delle loro librerie personali Steam, Ubisoft Connect, Origin\footnote{Origin è una piattaforma di distribuzione digitale di videogiochi sviluppata da Electronic Arts.}, Battle.net\footnote{Battle.net è una piattaforma di distribuzione digitale e di gestione dei diritti digitali sviluppata da Blizzard Entertainment.}\cite{Playkey}.


\subsubsection{Il caso Apple}
A metà del 2020 Apple aveva cercato di bloccare le app di cloud gaming sull'App Store, ma a settembre 2020 decise di consentire il cloud gaming con alcune restrizioni: che i giochi offerti nel servizio dovessero essere scaricati direttamente dall'App Store e non da un'app all-in-one. I produttori di app sono autorizzati a rilasciare una cosiddetta "app catalogo" che si collega ad altri giochi nel servizio, ma ogni gioco dovrà essere una singola app e tutti i giochi e le "app catalogo" devono offrire l'acquisto solo tramite il sistema di elaborazione dei pagamenti "in‑app purchases", in base al quale Apple di solito prende il 30\% delle entrate\cite{Apple_controversy}.
%license:BSD-3-Clause
%copyright-holders:Michele Maione
%============================================================
%
%	Piattaforma di cloud gaming per giochi arcade
%
%============================================================
\chapter{Tecnologie}

In questo capitolo 

\section{MPEG}
Lorem ipsum dolor sit amet, consectetur adipiscing elit, sed do eiusmod tempor incididunt ut labore et dolore magna aliqua. Ut enim ad minim veniam, quis nostrud exercitation ullamco laboris nisi ut aliquip ex ea commodo consequat. Duis aute irure dolor in reprehenderit in voluptate velit esse cillum dolore eu fugiat nulla pariatur. Excepteur sint occaecat cupidatat non proident, sunt in culpa qui officia deserunt mollit anim id est laborum.

\subsection{Compression}
Lorem ipsum dolor sit amet, consectetur adipiscing elit, sed do eiusmod tempor incididunt ut labore et dolore magna aliqua. Ut enim ad minim veniam, quis nostrud exercitation ullamco laboris nisi ut aliquip ex ea commodo consequat. Duis aute irure dolor in reprehenderit in voluptate velit esse cillum dolore eu fugiat nulla pariatur. Excepteur sint occaecat cupidatat non proident, sunt in culpa qui officia deserunt mollit anim id est laborum.

\subsection{Video}
Lorem ipsum dolor sit amet, consectetur adipiscing elit, sed do eiusmod tempor incididunt ut labore et dolore magna aliqua. Ut enim ad minim veniam, quis nostrud exercitation ullamco laboris nisi ut aliquip ex ea commodo consequat. Duis aute irure dolor in reprehenderit in voluptate velit esse cillum dolore eu fugiat nulla pariatur. Excepteur sint occaecat cupidatat non proident, sunt in culpa qui officia deserunt mollit anim id est laborum.

\subsection{Audio}
Lorem ipsum dolor sit amet, consectetur adipiscing elit, sed do eiusmod tempor incididunt ut labore et dolore magna aliqua. Ut enim ad minim veniam, quis nostrud exercitation ullamco laboris nisi ut aliquip ex ea commodo consequat. Duis aute irure dolor in reprehenderit in voluptate velit esse cillum dolore eu fugiat nulla pariatur. Excepteur sint occaecat cupidatat non proident, sunt in culpa qui officia deserunt mollit anim id est laborum.

\subsection{Trasmission}
Lorem ipsum dolor sit amet, consectetur adipiscing elit, sed do eiusmod tempor incididunt ut labore et dolore magna aliqua. Ut enim ad minim veniam, quis nostrud exercitation ullamco laboris nisi ut aliquip ex ea commodo consequat. Duis aute irure dolor in reprehenderit in voluptate velit esse cillum dolore eu fugiat nulla pariatur. Excepteur sint occaecat cupidatat non proident, sunt in culpa qui officia deserunt mollit anim id est laborum.



\section{FFmpeg}
Lorem ipsum dolor sit amet, consectetur adipiscing elit, sed do eiusmod tempor incididunt ut labore et dolore magna aliqua. Ut enim ad minim veniam, quis nostrud exercitation ullamco laboris nisi ut aliquip ex ea commodo consequat. Duis aute irure dolor in reprehenderit in voluptate velit esse cillum dolore eu fugiat nulla pariatur. Excepteur sint occaecat cupidatat non proident, sunt in culpa qui officia deserunt mollit anim id est laborum\cite{FFmpeg_Documentation}.

\subsection{Libs.}
Lorem ipsum dolor sit amet, consectetur adipiscing elit, sed do eiusmod tempor incididunt ut labore et dolore magna aliqua. Ut enim ad minim veniam, quis nostrud exercitation ullamco laboris nisi ut aliquip ex ea commodo consequat. Duis aute irure dolor in reprehenderit in voluptate velit esse cillum dolore eu fugiat nulla pariatur. Excepteur sint occaecat cupidatat non proident, sunt in culpa qui officia deserunt mollit anim id est laborum.



\section{Simple DirectMedia Layer (SDL)}
SDL è una libreria multipiattaforma che fornisce accesso di basso livello ad audio, tastiera, mouse, gamepad, hardware 3D e framebuffer 2D. Come mostrato in Fig. \ref{fig:sdl} SDL è costruito sopra le API di visualizzazione video del sistema operativo (in arancione), librerie di rendering 3D (in verde) e librerie che si interfacciano alla scheda audio (in rosso)\cite{SDL_Wiki}.

\begin{figure}[H]
	\includegraphics[width=\linewidth]{immagini/sdl}
	\caption{SDL: livelli di astrazione su diverse piattaforme}
	\label{fig:sdl}
\end{figure}


\subsection{Video}
Il MAME è in grado di emulare giochi sia 2D che 3D (es.: Tekken della Namco), ma poiché emula il monitor fisico ciò che viene inviato alle varie librerie grafiche è un insieme di primitive e texture da disegnare.

\begin{figure}[H]
	\includegraphics[width=\linewidth]{immagini/rendering_pipeline}
	\caption{Pipeline di rendering 2D}
	\label{fig:rendering_pipeline}
\end{figure}

Quando la finestra di gioco viene inizializzata, viene creato un contesto di rendering SDL per la finestra tramite la funzione \textit{CreateRenderer}. Per ogni frame della macchina che viene emulato c'è una fase di disegno usando \textit{SetRenderDrawColor}, \textit{RenderFillRect} e \textit{RenderDrawLine}, e poi tramite la funzione \textit{RenderPresent} viene mostrato il frame appena renderizzato sulla finestra.


\subsection{Audio}
Lorem ipsum dolor sit amet, consectetur adipiscing elit, sed do eiusmod tempor incididunt ut labore et dolore magna aliqua. Ut enim ad minim veniam, quis nostrud exercitation ullamco laboris nisi ut aliquip ex ea commodo consequat. Duis aute irure dolor in reprehenderit in voluptate velit esse cillum dolore eu fugiat nulla pariatur. Excepteur sint occaecat cupidatat non proident, sunt in culpa qui officia deserunt mollit anim id est laborum.



\section{Web APIs}
Le API Web sono un insieme di API e interfacce che comprendono la potente capacità di creazione di script del Web. A seguire quelli utilizzati in questo progetto\cite{Web_APIs}.

\subsection{WebSocket}
WebSocket è un protocollo di comunicazione del computer che fornisce canali di comunicazione full-duplex su una singola connessione TCP. È compatibile con HTTP perché l'handshake WebSocket utilizza l'intestazione di aggiornamento HTTP per passare dal protocollo HTTP al protocollo WebSocket. È supportato nativamente da tutti i browser e il suo utilizzo è simile ai normali socket sia sul lato client che su quello server. Per questi motivi è il protocollo di comunicazione generico più utilizzato sul web\cite{WebSocket_Web_APIs}.

\subsection{Canvas API}
L'API Canvas fornisce un mezzo per disegnare grafica tramite JavaScript, si concentra principalmente sulla grafica 2D ma quando viene utilizzata dall'API WebGL può disegnare grafica 2D e 3D con accelerazione hardware. È completamente supportato da tutti i browser\cite{Canvas_API}.

\subsection{WebGL API}
WebGL è un'API JavaScript, progettata e gestita dal gruppo no-profit Khronos, per il rendering di grafica 2D e 3D che consente l'utilizzo accelerato dalla GPU della fisica e dell'elaborazione e degli effetti delle immagini. WebGL 1.0 è supportato su tutti i browser, mentre WebGL 2.0 viene testato su Safari\cite{WebGL}.



\section{JavaScript libraries}
Per il front-end, sono state utilizzate tre librerie JavaScript open source per la gestione degli input e per la decodifica del filmato.

\subsection{JSMpeg}
JSMpeg è una libreria composta da un demuxer MPEG-TS, un decoder video MPEG1 e audio MP2, con un sistema di rendering basato sia su WebGL che su Canvas2D, ed un sistema di output audio basato su WebAudio. Consente lo streaming a bassa latenza ($\sim$50ms) tramite WebSocket, ed è rilasciata con licenza MIT\cite{JSMpeg}.

\subsection{Keypress}
Keypress è una libreria per la cattura dell'input da tastiera specializzata per l'uso in contesti videoludici, rilasciata con licenza Apache 2.0. Viene utilizzata per gestire l'input da tastiera nel front-end\cite{Keypress}.

\subsection{GameController.js}
GameController.js è una libreria che estende le Web API per il gamepad, è rilasciata con licenza MIT. Nel front-end viene utilizzata per gestire i gamepads, per consentire il multiplayer sullo stesso dispositivo\cite{gameController_js}.
%license:BSD-3-Clause
%copyright-holders:Michele Maione
%============================================================
%
%	Cloud gaming platform for arcade games
%
%============================================================

\chapter{System architecture}

Lorem ipsum dolor sit amet, consectetur adipiscing elit, sed do eiusmod tempor incididunt ut labore et dolore magna aliqua. Ut enim ad minim veniam, quis nostrud exercitation ullamco laboris nisi ut aliquip ex ea commodo consequat. Duis aute irure dolor in reprehenderit in voluptate velit esse cillum dolore eu fugiat nulla pariatur. Excepteur sint occaecat cupidatat non proident, sunt in culpa qui officia deserunt mollit anim id est laborum.

\section{MAME}
Lorem ipsum dolor sit amet, consectetur adipiscing elit, sed do eiusmod tempor incididunt ut labore et dolore magna aliqua. Ut enim ad minim veniam, quis nostrud exercitation ullamco laboris nisi ut aliquip ex ea commodo consequat. Duis aute irure dolor in reprehenderit in voluptate velit esse cillum dolore eu fugiat nulla pariatur. Excepteur sint occaecat cupidatat non proident, sunt in culpa qui officia deserunt mollit anim id est laborum.

\subsection{Libs.}
Lorem ipsum dolor sit amet, consectetur adipiscing elit, sed do eiusmod tempor incididunt ut labore et dolore magna aliqua. Ut enim ad minim veniam, quis nostrud exercitation ullamco laboris nisi ut aliquip ex ea commodo consequat. Duis aute irure dolor in reprehenderit in voluptate velit esse cillum dolore eu fugiat nulla pariatur. Excepteur sint occaecat cupidatat non proident, sunt in culpa qui officia deserunt mollit anim id est laborum.

\subsection{Server}
Lorem ipsum dolor sit amet, consectetur adipiscing elit, sed do eiusmod tempor incididunt ut labore et dolore magna aliqua. Ut enim ad minim veniam, quis nostrud exercitation ullamco laboris nisi ut aliquip ex ea commodo consequat. Duis aute irure dolor in reprehenderit in voluptate velit esse cillum dolore eu fugiat nulla pariatur. Excepteur sint occaecat cupidatat non proident, sunt in culpa qui officia deserunt mollit anim id est laborum.

\subsection{SDL renderer}
Lorem ipsum dolor sit amet, consectetur adipiscing elit, sed do eiusmod tempor incididunt ut labore et dolore magna aliqua. Ut enim ad minim veniam, quis nostrud exercitation ullamco laboris nisi ut aliquip ex ea commodo consequat. Duis aute irure dolor in reprehenderit in voluptate velit esse cillum dolore eu fugiat nulla pariatur. Excepteur sint occaecat cupidatat non proident, sunt in culpa qui officia deserunt mollit anim id est laborum.

\subsection{SDL audiomixer}
Lorem ipsum dolor sit amet, consectetur adipiscing elit, sed do eiusmod tempor incididunt ut labore et dolore magna aliqua. Ut enim ad minim veniam, quis nostrud exercitation ullamco laboris nisi ut aliquip ex ea commodo consequat. Duis aute irure dolor in reprehenderit in voluptate velit esse cillum dolore eu fugiat nulla pariatur. Excepteur sint occaecat cupidatat non proident, sunt in culpa qui officia deserunt mollit anim id est laborum.

\subsection{WebSocket implementation}
Lorem ipsum dolor sit amet, consectetur adipiscing elit, sed do eiusmod tempor incididunt ut labore et dolore magna aliqua. Ut enim ad minim veniam, quis nostrud exercitation ullamco laboris nisi ut aliquip ex ea commodo consequat. Duis aute irure dolor in reprehenderit in voluptate velit esse cillum dolore eu fugiat nulla pariatur. Excepteur sint occaecat cupidatat non proident, sunt in culpa qui officia deserunt mollit anim id est laborum.

\subsection{Encoder}
Lorem ipsum dolor sit amet, consectetur adipiscing elit, sed do eiusmod tempor incididunt ut labore et dolore magna aliqua. Ut enim ad minim veniam, quis nostrud exercitation ullamco laboris nisi ut aliquip ex ea commodo consequat. Duis aute irure dolor in reprehenderit in voluptate velit esse cillum dolore eu fugiat nulla pariatur. Excepteur sint occaecat cupidatat non proident, sunt in culpa qui officia deserunt mollit anim id est laborum.

\cite{CPP_Primer}
\cite{Computer_Networking_and_the_Internet}
\cite{Ingegneria_del_software}
\cite{Understanding_the_Linux_Kernel}
\cite{Windows_Server_2012}
%license:BSD-3-Clause
%copyright-holders:Michele Maione
%============================================================
%
%	Piattaforma di cloud gaming per giochi arcade
%
%============================================================

\chapter{Prestazioni} \label{cap:cap4}
%Si mostra il progetto dal punto di vista sperimentale, le cose materialmente realizzate. In questa sezione si mostrano le attività sperimentali svolte, si illustra il funzionamento del sistema (a grandi linee) e si spiegano i risultati ottenuti con la loro valutazione critica. Bisogna introdurre dati sulla complessità degli algoritmi e valutare l'efficienza del sistema.
Questo capitolo analizza le prestazioni del progetto relativamente ai tre difetti intrinseci del cloud gaming: riduzione della qualità audiovisiva, il problema della latenza ed il bit-rate richiesto.



\section{Riduzione della qualità audiovisiva}
Nell'ambito videoludico la qualità audiovisiva (soprattutto la visiva) influisce sulla qualità generale percepita dall'utente, per questo nel cloud gaming è necessario garantire un livello di qualità quasi identico alla versione in locale. Lo streaming a bit-rate bassi è possibile preservando la qualità percepita ed è particolarmente adatto per l'utilizzo in reti mobili\footnote{Con l'utilizzo di tariffe flat per reti mobili. Solitamente sono abbonamenti solo dati per tablet e smartphone.} e WLAN \parencite{VideoAndMultimediaTransmissionsOverCellularNetworks}.



\subsection{La qualità visiva}
Qualsiasi elaborazione applicata ad un'immagine può causare la perdita di qualità. La qualità dell'immagine si riferisce al modo in cui viene riprodotta la scena ripresa (o quella generata digitalmente) e può essere descritta dal degradarsi di alcune caratteristiche tra cui: nitidezza, gamma dinamica, fusione dei colori, contrasto, sfocatura, ecc\dots. I metodi di valutazione della qualità dell'immagine possono essere suddivisi in metodi oggettivi e soggettivi. I metodi oggettivi si basano su confronti utilizzando criteri numerici espliciti. L'errore quadratico medio (MSE), il \textit{peak signal-to-noise ratio} (PSNR) e la \textit{structural similarity index measure} (SSIM) sono esempi di misure di qualità utilizzate nell'analisi delle immagini \parencite{relationship_PSNR_and_SSI}.



\subsubsection{Peak signal-to-noise ratio}
Il PSNR è comunemente usato come misura della qualità della compressione e della riduzione del rumore nelle immagini digitali a scala di grigi. Esistono diversi approcci per calcolare il PSNR delle immagini a colori, ad esempio (il più semplice) utilizzando immagini in formato YCbCr e calcolando il PSNR solo sul canale della luminanza. Il PSNR è espresso solitamente in decibel con valori tipici per la compressione video per lo streaming tra $20$ \si{dB} e $30$ \si{dB} \parencite{ThomosN2006OtoJ} e si calcola utilizzando l'Eq. \ref{eq:PSNR} con $MSE(f,g)$ l'errore quadratico medio e $L = 255$ l'intervallo dinamico \parencite{AnewcombinedPSNRforobjectivevideoqualityassessment}.

\begin{equation} \label{eq:PSNR}
	PSNR(f,g)=10 \log_{10}  \frac{L^2}{MSE(f,g)}	
\end{equation}

\begin{figure}[H]
	\centering
	\includegraphics[width=\linewidth]{immagini/BMP_MPEG_compare}
	\caption{Alcuni frame d'esempio: a sinistra i frame renderizzati dal MAME (RGBA), a destra i frame codificati in \textit{MPEG-1}. © Capcom, Data East, Taito}
	\label{fig:BMP_MPEG_compare}
\end{figure}



\subsubsection{Structural Similarity Index Measure}
SSIM è un metodo per predirre la qualità percepità nelle immagini digitali. Invece di utilizzare i tradizionali metodi di somma degli errori, SSIM è progettato modellando qualsiasi distorsione dell'immagine come una combinazione di tre fattori: la perdita di correlazione, la distorsione della luminanza e la distorsione del contrasto. L'indice SSIM è un valore decimale compreso tra $0$ e $1$ e si calcola utilizzando l'Eq. \ref{eq:SSIM} con \parencite{relationship_PSNR_and_SSI}:

\begin{itemize}
	\item $k_1 = 0,01$ e $k_2 = 0,03$ costanti;
	\item $L = 255$ l'intervallo dinamico;
	\item $c_1 = (k_1 L)^2$, $c_2 = (k_2 L)^2$ e $c_3 = c_2/2$ variabili di stabilizzazione della divisione;
	\item $l(f,g)$ funzione di comparazione della luminanza tra le medie $\mu_f$ e $\mu_g$;
	\item $c(f,g)$ funzione di comparazione del contrasto tra le deviazioni standard $\sigma_f$ e $\sigma_g$;
	\item $s(f,g)$ funzione di comparazione della struttura che misura il coefficiente di correlazione; $\sigma_{fg}$ è la covarianza tra $f$ e $g$.
\end{itemize}

\begin{equation} \label{eq:SSIM}
	\begin{matrix*}[l]
		SSIM(f,g) =  l(f,g) c(f,g) s(f,g) & l(f,g)= \frac{2 \mu_f \mu_g  + C_1}{\mu_f^2 \mu_g^2  + C_1} \\
		c(f,g)= \frac{2 \sigma_f \sigma_g  + C_2}{\sigma_f^2 + \sigma_g^2  + C_2} & s(f,g)= \frac{\sigma_{fg}+C_3}{\sigma_f \sigma_g + C_3}
	\end{matrix*}
\end{equation}



\subsubsection{Risultati ottenuti}
Tramite lo script in Lis. \ref{lst:PyPSNR_SSIM} è stata calcolata la qualità video per la codifica \textit{MPEG-1 Video} a 1024 kbps che ha dato come risultati un valore di PSNR di $27,703$ \si{dB} e un indice SSIM del $89,9\%$. Alcune immagini d'esempio del risultato ottenuto sono visibili in Fig. \ref{fig:BMP_MPEG_compare}.

\begin{lstlisting}[language=Python, caption=Script Python per il calcolo di PSNR e SSIM, label={lst:PyPSNR_SSIM}]
imgRGB = skimage.io.imread("frameRGB.bmp")
imgMPEG1 = skimage.io.imread("frameMPEG1.bmp")

PSNR = skimage.metrics.peak_signal_noise_ratio(imgRGB, imgMPEG1)

SSIM = skimage.metrics.structural_similarity(imgRGB, imgMPEG1,
	multichannel=True)
\end{lstlisting}



\subsection{La qualità uditiva}
Gli algoritmi sviluppati per misurare oggettivamente la qualità audio percepita simulano le proprietà percettive dell'orecchio umano ed applicano modelli informatici per stimare le similarità tra due segnali audio. Tra questi algoritmi abbiamo \textit{Perceptual Evaluation of Audio Quality} (PEAQ), \textit{Perceptual Objective Listening Quality Assessment} (POLQA) ed \textit{Evaluation of Audio Quality} (EAQUAL). Questi algoritmi sono brevettati e protetti da licenza. Per PEAQ è disponibile un'implementazione per uso accademico a cura di Giuseppe Gottardi\footnote{Il repository del progetto è disponibile qui: sourceforge.net/projects/peaqb.} \parencite{PoctaPeter2015SaOA}.



\subsubsection{Perceptual Evaluation of Audio Quality}
\textit{Perceptual Evaluation of Audio Quality} (PEAQ) è un algoritmo sviluppato dal ITU-R\footnote{ITU-R: \textit{International Telecommunication Union - Radiocommunication Sector} è un organizzazione internazionale che si occupa degli standard nel campo delle telecomunicazioni, settore radio.} che simula le proprietà percettive dell'orecchio umano tramite modelli informatici, per determinare il livello di rumore che si può introdurre in un segnale acustico prima che esso diventi udibile. Per PEAQ sono stati creati due modelli, \textit{basic} e \textit{advanced}, il modello avanzato viene utilizzato per test più approfonditi \parencite{UlovecK2018PAQA}. Gli output dell'algoritmo sono il \textit{Objective Difference Grade} (ODG) e l'\textit{indice di distorsione} (DI) che vengono assegnati ad ogni frame audio, in questo modo si può intervenire precisamente sull'algoritmo di codifica per migliorarlo. Facendo la media degli ODG di tutti i frame si può dare una stima della qualità di tutto il segnale. La scala ODG è basata su un test uditivo soggettivo (ITU-R BS.1116) ed è la seguente: \textit{molto fastidioso} (1), \textit{fastidioso} (2), \textit{leggermente fastidioso} (3), \textit{percettibile ma non fastidioso} (4), \textit{impercettibile} (5) \parencite{PoctaPeter2015SaOA}.

Il metodo, illustrato in Fig. \ref{fig:PEAQ}, inizia applicando un allineamento temporale tra i due segnali in ingresso (alcuni codificatori slittano leggermente il segnale in avanti), questi vengono inviati al modello percettivo. I segnali audio vengono tagliati in frame di $0,042$ s con una sovrapposizione del $50\%$. Il modello percettivo trasforma i segnali utilizzando una funzione finestra e una \textit{FFT}. Successivamente vengono applicati dei pre-processamenti per calcolare: i modelli di eccitazione, i modelli di volume, i modelli di modulazione e il segnale di errore. Utilizzando questi modelli una funzione calcola le \textit{Model Output Variables} (MOVs) che vengono mappate sul corrispettivo \textit{ODG} \parencite{PEAQ}.

\begin{figure}[H]
	\includegraphics[width=\linewidth]{immagini/PEAQ}
	\caption{Diagramma del funzionamento di PEAQ}	
	\label{fig:PEAQ}
\end{figure}



\subsubsection{Risultati ottenuti}
In questo progetto è stata usata un'implementazione per scopi accademici di PEAQ e tramite lo script bash mostrato in Lis. \ref{lst:bashPEAQ} è stato calcolato per la codifica \textit{MPEG-1 Audio Layer II} (MP2) a 64 kbps un grado PEAQ di 2 (\textit{fastidioso}).

\begin{lstlisting}[language=bash, caption=Script bash per l'esecuzione del programma per l'algoritmo PEAQ, label={lst:bashPEAQ}]
peaqb -r sf3_PCM.wav -t sf3_MP2.wav -c > risultati.csv
\end{lstlisting}




\section{Il problema della latenza} \label{sec:cap4_Latenza}
Alle fasi di base di un videogioco (ricezione input, esecuzione, rendering e display) nel cloud gaming vanno ad aggiungersi: invio al server dell'input utente, codifica, trasmissione e decodifica. Queste fasi aggiuntive aumentano il tempo che intercorre tra l'input del giocatore e l'azione visualizzata sullo schermo.

Un leggero ritardo in un film in streaming o in una videochiamata molto probabilmente passa inosservato, ma durante una partita la latenza può rendere il gioco ingiocabile, fortunatamente il rapido sviluppo delle reti a banda larga hanno reso questo problema meno evidente e il cloud gaming una realtà.

Secondo diversi studi, affinché il gameplay sia fluido e senza jitter\footnote{Il jitter è la variazione del ritardo tra i pacchetti inviati.}, il ritardo tra la trasmissione dell'input utente e la visualizzazione dell'azione corrispondente, chiamato anche round trip delay, deve essere inferiore a 100 millisecondi. Se il round trip delay è maggiore della soglia di ritardo tollerata, la reazione dell'utente viene visualizzata con un ritardo che rende il gameplay difficoltoso e irritante, soprattutto nei giochi in cui la velocità è un fattore molto importante, come gli sparatutto in prima persona e i giochi di strategia in tempo reale. In tabella \ref{table:Ritardo_tollerato_per_tipo_di_gioco} sono riportate le soglie di ritardo tollerabili per tipo di gioco \parencite{Cloud_Gaming_Architecture_and_Performance}.

\begin{table}[H]
	\centering
	\begin{tabular}{||l l r||}
		\hline
		Tipo di gioco & Prospettiva & Soglia di ritardo (ms) \\
		\hline\hline
		FPS & Prima persona & 100 \\
		\hline
		RPG & Terza persona & 500 \\
		\hline
		RTS & Onnipresente & 1000 \\
		\hline
	\end{tabular}

	\caption{Ritardo tollerato per tipo di gioco}
	\label{table:Ritardo_tollerato_per_tipo_di_gioco}
\end{table}



\subsubsection{Risultati ottenuti}
La latenza è stata calcolata cronometrando il \textit{round trip delay} e calcolando le medie al termine della partita, lato server della codifica\footnote{Il tempo medio di codifica contiene anche il tempo di cattura.} e della creazione del pacchetto di rete, e lato client della decodifica. Per calcolare il \textit{round trip delay} è stata eseguita una sessione di gioco di \textit{Street Fighter III} ed è stato effettuato lo screencast utilizzando il programma \textit{OBS Studio}\footnote{OBS Studio è un software open-source per lo streaming e la cattura video.} a 60 fps. Utilizzando una \textit{tastiera su schermo} veniva premuto il tasto \textit{P} per mettere il gioco in pausa, successivamente è stato analizzato manualmente lo screencast calcolando il \textit{round trip delay} medio tra il frame in cui avveniva la pressione del tasto e il frame del gioco in pausa. Il tempo medio di trasmissione\footnote{Il tempo medio di trasmissione contiene anche il tempo di visualizzazione del frame sullo schermo.} è stato calcolato come la differenza tra il \textit{round trip delay} medio e le altre operazioni. I tempi medi ottenuti sono mostrati in Tab. \ref{table:LatenzaOttenuta}.

Le caratteristiche dei computer utilizzati per i test sono descritte nelle Tab. \ref{table:ServerUsato} e \ref{table:ClientUsato}.

\begin{table}[H]
	\centering
	\begin{tabular}{||l r||}
		\hline
		Operazione & Tempo (ms) \\
		\hline\hline				
		\hline
		Codifica & $6,71$ \\
		\hline
		Creazione pacchetto di rete & $0,07$ \\
		\hline
		Trasmissione & $\sim 174,00$ \\
		\hline
		Decodifica & $1,04$ \\
		\hline\hline
		\textbf{Round trip delay} & \textbf{$\sim 182,00$} \\
		\hline
	\end{tabular}

	\caption{Tempi medi d'elaborazione per \textit{Street Fighter III}}
	\label{table:LatenzaOttenuta}
\end{table}

\begin{table}[H]
	\centering
	\begin{tabular}{||l l||}
		\hline
		Componente & Modello \\
		\hline\hline				
		\hline
		CPU & AMD Ryzen 5 2500U (8 threads) @ $3,6$ GHz \\
		\hline
		GPU & AMD Radeon Vega 8 Mobile Graphics $4$ GB \\
		\hline
		RAM & 8 GB \\
		\hline
		Disco & SanDisk SSD SD9SN8W \\
		\hline
		S.O. & Fedora 34 \\
		\hline
	\end{tabular}

	\caption{Server utilizzato per il test}
	\label{table:ServerUsato}
\end{table}


\begin{table}[H]
	\centering
	\begin{tabular}{||l l||}
		\hline
		Componente & Modello \\
		\hline\hline				
		\hline
		CPU & Intel Core2 Quad Q6600 (4 threads) @ $2,4$ GHz \\
		\hline
		GPU & Nvidia GeForce 9500 GT 1 GB\\
		\hline
		RAM & 8 GB \\
		\hline
		Disco & Samsung 860 EVO SSD \\
		\hline
		S.O. & Windows 10 \\
		\hline
		Browser & Firefox 88 \\
		\hline
	\end{tabular}

	\caption{Client utilizzato per il test}
	\label{table:ClientUsato}
\end{table}


\section{Bit-rate richiesto}
Nel paradigma del cloud gaming lo streaming viene effettuato in tempo reale e non c'è possibilità di utilizzare alcun tipo di buffer, per questo motivo è necessario stimare il bit-rate richiesto per usufruire del servizio senza \parencite{Network_Analysis_of_the_Sony_Remote_Play_System}. Ciao \parencite{StreamingMobileCloudGamingVideoOverTCPWithAdaptiveSourceFECCoding}. Probabilmente \parencite{ChenKuanTa2014OtQo}.

\begin{table}[H]
	\centering
	\begin{tabular}{||l l r r r||}
		\hline
		Gioco & Tipo & Media & Dev. std & Varianza \\
		\hline\hline
		\hline
		Castelvania & piattaforma & 0.867 & 0.537 & 0.289 \\
		\hline		
		Metal Slug 3 & piattaforma & 0.688 & 0.360 & 0.129 \\
		\hline
		Out Run & guida & 0.835 & 0.443 & 0.196 \\
		\hline
		Puzzle Booble 2 & puzzle & 0.731 & 0.320 & 0.102 \\
		\hline
		Street Fighter III & picchiaduro & 0.848 & 0.400 & 0.160 \\
		\hline	
		Street Slam & sport & 1.075 & 0.555 & 0.308 \\
		\hline
	\end{tabular}

	\caption{Client utilizzato per il test}
	\label{table:bandwith_e_medie}
\end{table}

\begin{figure}
	\includegraphics[width=\linewidth]{immagini/bandwith}
	\caption{Bit-rate richiesto per vari tipi di giochi}	
	\label{fig:bandwith}
\end{figure}
%license:BSD-3-Clause
%copyright-holders:Michele Maione
%============================================================
%
%	Piattaforma di cloud gaming per giochi arcade
%
%============================================================

\chapter*{Direzioni future di ricerca e conclusioni}
\addcontentsline{toc}{chapter}{Direzioni future di ricerca e conclusioni}

%Si mostrano le prospettive future di ricerca nell'area dove si è svolto il lavoro. Talvolta questa sezione può essere l'ultima sottosezione della precedente. Nelle conclusioni si deve richiamare l'area, lo scopo della tesi, cosa è stato fatto, come si valuta quello che si è fatto e si enfatizzano le prospettive future per mostrare come andare avanti nell'area di studio.

Durante la realizzazione di questo progetto molte persone l'hanno provato e hanno chiesto quando sarebbe stato reso pubblico per poterlo utilizzare, ed è evidente il grande interesse che hanno ancora oggi i videogiochi arcade per le persone di tutte le età. Il \textit{MAME CGP} si prefiggeva lo scopo di rendere fruibili nel modo più semplice possibile e per la maggior parte dei dispositivi i videogiochi arcade, scopo che si è concretizzato sottoforma di piattaforma di cloud gaming. Credo fermamente che il rilascio del codice sorgente del \textit{MAME CGP} stimolerà altri sviluppatori ad una maggiore innovazione sui sistemi di cloud gaming e sulle tecnologie di streaming in generale, nonché ad ampliamenti del progetto.

Il progetto apre le porte a molte idee e miglioramenti che possono portarlo ad essere effettivamente competitivo. Molto utile per la fidelizzazione dell'utente sarebbe la creazione di un sistema di account tramite cui il giocatore può salvare e caricare lo stato del gioco, pubblicare i punteggi nella leaderboard, invitare altri giocatori ad unirsi alla partita supportando così il multiplayer da devices differenti. Per ridurre la dimensione dei pacchetti si potrebbero usare due codec open-source che in futuro saranno supportati nativamente dalla maggior parte dei browser: \textit{AOMedia Video 1 (AV1)}, progettato per trasmissioni video su Internet, e \textit{Opus}, un codec audio lossy utilizzato per la comunicazione in tempo reale. Ambedue inseribili nel contenitore \textit{WebM}. Per diminuire l'overhead di comunicazione si potrebbe sostituire il protocollo \textit{WebSocket} con \textit{RTP} utilizzabile tramite la tecnologia \textit{WebRTC}. Nella sua versione attuale il progetto è in ascolto su una porta specifica e può essere distribuito su più server utilizzando un software per il bilanciamento del carico (ad esempio \textit{Nginx}), oppure ampliando la classe che si occupa di orchestrare la comunicazione con il client, sfruttando lo stesso protocollo utilizzato per la ricezione dell'input utente si possono scambiare informazioni con altre instanze del \textit{MAME CGP} in esecuzione su altri server per un'efficiente bilanciamento del carico. Un'altra strategia di ottimizzazione, sebbene complessa, consiste nel dividere le varie fasi del cloud gaming in componenti e distribuirle dinamicamente su più server in base alle risorse richieste e disponibili.

A valle di tutto quello di cui si è parlato finora si ritiene che gli obiettivi iniziali della tesi siano stati raggiunti e che il \textit{MAME CGP} è in grado di fungere da piattaforma di cloud gaming. Infine, il fatto che gli sviluppi possibili siano molteplici induce a pensare che il sistema abbia ampi margini evolutivi.

\appendix
\newpage
\pagenumbering{roman}
\setcounter{page}{\thesavepage}

\chapter{Documentazione del progetto logico}
Documentazione del progetto logico dove si documenta il progetto logico del sistema e se è il caso si mostra la progettazione in grande del SW e dell’HW. Quest’appendice mostra l’architettura logica implementativa (nella Sezione 4 c’era la descrizione, qui ci vanno gli schemi a blocchi e i diagrammi).

\chapter{Documentazione della programmazione}
Documentazione della programmazione in piccolo dove si mostra la struttura ed eventualmente l’albero di Jackson.

\chapter{Listato}
Il listato (o solo parti rilevanti di questo, se risulta particolarmente esteso) con l’autodocumentazione relativa.

\chapter{Manuale utente}
Manuale utente per l’utilizzo del sistema

\chapter{Esempio di impiego}
Un esempio di impiego del sistema realizzato.

\chapter{Datasheet}
Eventuali Datasheet di riferimento.

\backmatter
\printbibliography[nottype=misc,title={Bibliografia},heading=bibintoc]
\printbibliography[type=misc,title={Sitografia},heading=bibintoc]

\newpage
\pagenumbering{gobble}
%license:BSD-3-Clause
%copyright-holders:Michele Maione
%============================================================
%
%	Piattaforma di cloud gaming per giochi arcade
%
%============================================================

\chapter*{Ringraziamenti 2.0}

Ringrazio tutti coloro che hanno fatto parte del mio percorso di laurea magistrale:\\
La mia famiglia che crede sempre che io possa fare tutto, senza capire che nei miei limiti è sempre una faticaccia.\\
Dino che ha chiuso il buco che avevo nel petto.\\
Laura \& Giulia, Simona, Eleonora, Martina, Greta e tutte le altre ragazze del dipartimento di farmacia, grazie per aver reso divertenti le giornate di studio, c'era sempre il sole in biblioteca.\\
I giocatori del BawiTeam. Tra lacrime, infortuni, risate e gioie. Che squadra meravigliosa!\\
I miei compagni di corso: Carrarini, Dettori, Iervolino, Lombardi, Bonapace, Paduano, Vannucci, Zhab'yak per i fantastici progetti fatti insieme.\\
Mario, Fede, Nadia, Giorgio, Giovanni e Mariapina per esserci sempre stati (da oltre 20 anni!).\\
Alessandro, Carmen, Claudio, Barbara, Marika, Grazia, Emiliana, Kikka, anche se non ci siamo visti spesso siete stati vicini.\\
I miei parenti di Treviglio che mi hanno aiutato e ospitato, grazie per il supporto.\\
Tutti gli altri, anche se non menzionati, siete nel mio cuore.

\vspace*{\fill}

\begin{figure}[H]
	\centering
	\includegraphics[width=5cm]{immagini/hadoken}
	\caption{Street Fighter Alpha artwork. © Capcom}
	\label{fig:hadoken}
\end{figure}

\end{document}