% !TeX TXS-program:compile = txs:///pdflatex/[--shell-escape]

\documentclass[12pt,british]{report}
\usepackage[T1]{fontenc}
\usepackage{tesi}
\usepackage{csharp_algorithms}

\def\myCDL{Master's degree program in Computer Science}
\def\myTitle{Cloud gaming platform for arcade games}

\def\myName{Michele Maione}
\def\myMat{Matr. No. 931468}

\def\myRefereeA{Prof. Dario Maggiorini}
\def\myRefereeB{Prof. Davide Gadia}

\def\myYY{2020-2021}

% Il seguente comando introduce un elenco delle figure dopo l'indice (facoltativo)
%\figurespagetrue

% Il seguente comando introduce un elenco delle tabelle dopo l'indice (facoltativo)
%\tablespagetrue

%
%			PREAMBOLO
%			Inserire qui eventuali package da includere o definizioni di comandi personalizzati
%

% Package di formato
\usepackage[
%    top    = 2.75cm,
%    bottom = 2.50cm,
%    left   = 3.00cm,
%    right  = 3.00cm,
    a4paper]{geometry}				% Formato del foglio
\usepackage[british]{babel}			% Supporto per l'inglese
\usepackage[utf8]{inputenc}			% Supporto per UTF-8
%\usepackage[a-1b]{pdfx}			% File conforme allo standard PDF-A (obbligatorio per la consegna)
\usepackage{float}

% Package per la grafica
\usepackage{graphicx}				% Funzioni avanzate per le immagini
\usepackage{hologo}					% Bibtex logo with \hologo{BibTeX}
%\usepackage{epsfig}				% Permette immagini in EPS
%\usepackage{xcolor}				% Gestione avanzata dei colori

% Package tipografici
\usepackage{amssymb,amsmath,amsthm} % Simboli matematici
\usepackage{listings}				% Scrittura di codice

% Package ipertesto
\usepackage{url}					% Visualizza e rendere interattii gli URL
\usepackage{hyperref}				% Rende interattivi i collegamenti interni
\usepackage{cleveref}

\usepackage{algorithm} 
\usepackage{algpseudocode} 

\usepackage{verbatim}
\usepackage{commath}
\usepackage{minted}

\begin{document}
\renewcommand\contentsname{Index}
% Creazione automatica del frontespizio
\frontespizio
\beforepreface

% PAGINA DI DEDICA E/O CITAZIONE
{
\raggedleft \large \sl To the new generations.\\
	
	\vspace{2cm}
	
	``Se non così, come?\\E se non ora, quando?''
	
	\bigskip
	
	\--- Primo Levi\\
}
         
% 
%			PREFAZIONE (facoltativa)
%

%Le prefazioni non sono molto comuni, tuttavia a volte capita che qualcuno voglia 
\prefacesection{Abstract}
In 1952 in the laboratories of the University of Cambridge A.S. Douglas, as an example for his doctoral thesis, created OXO, the transposition of the Tic-Tac-Toe computer game. This is usually considered technically the first video game as it used a cathode screen for display [5]. Its purpose, however, was not to entertain users but to complete Douglas's thesis. In 1958 the physicist Willy Higinbotham of Brookhaven National Laboratory, noting the lack of interest that students had in the subject, created a game, Tennis for Two, which had the task of simulating the physical laws that could be found in a tennis match: the medium used was an oscilloscope. This is remembered as a university experiment rather than a game.
In 1961, six young scientists from MIT (Massachusetts Institute of Technology) manage to give movement to luminous dots on the screen of a PDP-1: Spacewar! Was born, the first video game properly designed for recreational purposes that history remembers.
Two months later Nolan Bushnell and Ted Dabney completed their version of Spacewar !, called Computer Space and built by Nutting Associates. Considered the first large-scale arcade video game, around 1,500 were produced; however, the game was not a great success due to the high difficulty.
Bushnell, after the not particularly successful experiment of Computer Space, still decided to insist with video games, but producing them on his own: thus Atari was born. Atari's first arcade video game was the industry's first big hit: PONG, released on November 29, 1972, which roughly reproduces the mechanics of ping pong. Atari sold 19,000 Pong cabinets and soon many imitators followed suit. At the end of the decade the golden age of arcade video games began.
Having now become a mass cultural phenomenon, the video game is a unique medium: in fact, as James Paul Gee [6] suggests, video games are very different from other types of media (film, literature, theater ...), even if they incorporate the various languages. They have several characteristics that make them unique and operate differently from others, for example the language of gameplay is unique among traditional narrative media and it has also been authoritatively stated that it is interactivity that has distinguished video games from other forms of entertainment. mass media entertainment; precisely this characteristic allows the video game to exercise a potential of immersion and attraction that other media do not have.
The most conspicuous age range of those who play the video game is between 16 and 29 years, although in some countries, such as the United Kingdom, the average age is higher, with half of the total gamers over 40 years old\cite{High_Score}.

%
%			RINGRAZIAMENTI (facoltativi)
%

\prefacesection{Acknowledgements}
I would like to thank my supervisor Prof. Dario Maggiorini for his continuous support and Prof. Davide Gadia for his helpful guidance throughout my efforts towards the completion of the present thesis.

%
%			Creazione automatica dell'indice
%

\afterpreface

%			CAPITOLO 1: Introduzione 

\chapter{Introduction}
\label{cap:introduction}

Qui si possono vedere come funzionano le sezioni e le label, che possono essere referenziate così: \Cref{cap:introduction} o così se preferite la minuscola \cref{cap:introduction}.




contents


%			CAPITOLO 2: 
\chapter{System architecture}

Lorem ipsum dolor sit amet, consectetur adipiscing elit, sed do eiusmod tempor incididunt ut labore et dolore magna aliqua. Ut enim ad minim veniam, quis nostrud exercitation ullamco laboris nisi ut aliquip ex ea commodo consequat. Duis aute irure dolor in reprehenderit in voluptate velit esse cillum dolore eu fugiat nulla pariatur. Excepteur sint occaecat cupidatat non proident, sunt in culpa qui officia deserunt mollit anim id est laborum.

Lorem ipsum dolor sit amet, consectetur adipiscing elit, sed do eiusmod tempor incididunt ut labore et dolore magna aliqua. Ut enim ad minim veniam, quis nostrud exercitation ullamco laboris nisi ut aliquip ex ea commodo consequat. Duis aute irure dolor in reprehenderit in voluptate velit esse cillum dolore eu fugiat nulla pariatur. Excepteur sint occaecat cupidatat non proident, sunt in culpa qui officia deserunt mollit anim id est laborum.

Lorem ipsum dolor sit amet, consectetur adipiscing elit, sed do eiusmod tempor incididunt ut labore et dolore magna aliqua. Ut enim ad minim veniam, quis nostrud exercitation ullamco laboris nisi ut aliquip ex ea commodo consequat. Duis aute irure dolor in reprehenderit in voluptate velit esse cillum dolore eu fugiat nulla pariatur. Excepteur sint occaecat cupidatat non proident, sunt in culpa qui officia deserunt mollit anim id est laborum.

\cite{wikipedia}
\cite{CPP_Primer}
\cite{Computer_Networking_and_the_Internet}
\cite{Ingegneria_del_software}
\cite{Understanding_the_Linux_Kernel}
\cite{Windows_Server_2012}


%			CAPITOLO 3: 
\chapter{Elenchi}
\label{chap:elenchi}

Elenco puntato:
\begin{itemize}
    \item oggetto uno
    \item oggetto 2
\end{itemize}
\noindent
Elenco numerato:
\begin{enumerate}
    \item oggetto 1
    \item oggetto 2
\end{enumerate}
\noindent
Elenco puntato:
\begin{itemize}
    \item oggetto uno
    \item 
    Elenco puntato:
    \begin{itemize}
        \item oggetto uno
        \item oggetto 2
    \end{itemize}
\end{itemize}

%			CAPITOLO 4: 
\chapter{Immagini}


immagini, la H forza la posizione all'interno della pagina, se no la mette dove ci sta meglio come dimensioni, la larghezza va basata rispetto alla larghezza del testo:
\begin{figure}[H]
    \centering
    \includegraphics[width=0.8\textwidth]{immagini/unimi.pdf}
    \caption{Caption.}
    \label{fig:unimi}
\end{figure}




%			CAPITOLO 5: 
\chapter{Algorithm}

Algoritmi: cercare su overleaf per ulteriori istruzioni, esempio:

\begin{algorithm}
	\caption{PPO} 
	\begin{algorithmic}[1]
		\For {$iteration=1,2,\ldots$}
			\For {$actor=1,2,\ldots,N$}
				\State Run policy $\pi_{\theta_{old}}$ in environment for $T$ time steps
				\State Compute advantage estimates $\hat{A}_{1},\ldots,\hat{A}_{T}$
			\EndFor
			\State Optimize surrogate $L$ wrt. $\theta$, with $K$ epochs and minibatch size $M\leq NT$
			\State $\theta_{old}\leftarrow\theta$
		\EndFor
	\end{algorithmic} 
\end{algorithm}



Formula in C\# usando lstlisting, usa il file csharp\_algorithms.tex, ci sono due liste per fare l'highlight delle parole:
\\
\begin{minipage}{1\linewidth}
\begin{lstlisting}
//commento
public float Funzione(Vector3 p1, Vector3 p2)
{
    float v3 = p1+p2;
    return (1.0f / 6.0f) * v3;
}
\end{lstlisting}
\end{minipage}


%			CAPITOLO 6:
\chapter{Conclusions and future works}
\label{chap:conclusions}

\section{conclusions}
E qui finisce, spero che ci sia un po' tutto



%			Ringraziamenti:
%license:BSD-3-Clause
%copyright-holders:Michele Maione
%============================================================
%
%	Piattaforma di cloud gaming per giochi arcade
%
%============================================================

\chapter*{Ringraziamenti 2.0}

Ringrazio tutti coloro che hanno fatto parte del mio percorso di laurea magistrale:\\
La mia famiglia che crede sempre che io possa fare tutto, senza capire che nei miei limiti è sempre una faticaccia.\\
Dino che ha chiuso il buco che avevo nel petto.\\
Laura \& Giulia, Simona, Eleonora, Martina, Greta e tutte le altre ragazze del dipartimento di farmacia, grazie per aver reso divertenti le giornate di studio, c'era sempre il sole in biblioteca.\\
I giocatori del BawiTeam. Tra lacrime, infortuni, risate e gioie. Che squadra meravigliosa!\\
I miei compagni di corso: Carrarini, Dettori, Iervolino, Lombardi, Bonapace, Paduano, Vannucci, Zhab'yak per i fantastici progetti fatti insieme.\\
Mario, Fede, Nadia, Giorgio, Giovanni e Mariapina per esserci sempre stati (da oltre 20 anni!).\\
Alessandro, Carmen, Claudio, Barbara, Marika, Grazia, Emiliana, Kikka, anche se non ci siamo visti spesso siete stati vicini.\\
I miei parenti di Treviglio che mi hanno aiutato e ospitato, grazie per il supporto.\\
Tutti gli altri, anche se non menzionati, siete nel mio cuore.

\vspace*{\fill}

\begin{figure}[H]
	\centering
	\includegraphics[width=5cm]{immagini/hadoken}
	\caption{Street Fighter Alpha artwork. © Capcom}
	\label{fig:hadoken}
\end{figure}

% 
%			APENDICE: materiali aggiuntivi e dimostrazioni
% 

\appendix

\chapter{A}


%
%			BIBLIOGRAFIA
%

\bibliographystyle{unsrt}
\bibliography{bibliography}
\addcontentsline{toc}{chapter}{Bibliography}


\end{document}
