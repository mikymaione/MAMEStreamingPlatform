%license:BSD-3-Clause
%copyright-holders:Michele Maione
%============================================================
%
%	Piattaforma di cloud gaming per giochi arcade
%
%============================================================

\chapter{Architettura del sistema}
In questo capitolo verrà fatta\dots Lorem ipsum dolor sit amet, consectetur adipiscing elit, sed do eiusmod tempor incididunt ut labore et dolore magna aliqua. Ut enim ad minim veniam, quis nostrud exercitation ullamco laboris nisi ut aliquip ex ea commodo consequat. Duis aute irure dolor in reprehenderit in voluptate velit esse cillum dolore eu fugiat nulla pariatur. Excepteur sint occaecat cupidatat non proident, sunt in culpa qui officia deserunt mollit anim id est laborum.



\section{Sistema proposto}
L'esigenza per la quale nasce questo progetto è far conoscere alle nuove generazioni i videogiochi che hanno fatto la storia e dare la possibilità di poter giocare ancora a macchine che ormai hanno cessato di funzionare per motivi di obsolescenza, sfruttando due tecnologie entrate a far parte della quotidianità, lo streaming e il cloud computing. In questo lavoro si propone la creazione di una piattaforma di cloud gaming, che permette lo streaming audio-video direttamente e su richiesta dei videogiochi, da un server remoto, ad un client (computer, console, telefono). Per far ciò verrà ampliato il software MAME (rilasciato sotto licenza GNU-GPL) che è in grado di emulare oltre 7.000 giochi arcade. Le caratteristiche principali del progetto, che sono state vincolanti nella scelta delle tecnologie da utilizzare, sono la portabilità e la possibilità di utilizzare il sistema senza dover installare software aggiuntivi, e per questi vincoli, lato client, la scelta è ricaduta sul browser web. In Fig. \ref{fig:webprotocols} sono schematizzati i protocolli, i servizi e le API relativi alle tecnologie di streaming per i browser web ad oggi disponibili\cite{Audio_and_video_delivery}, che sono:

\begin{itemize}
	\item WebSocket è un protocollo di comunicazione che fornisce un canale full-duplex su una singola connessione TCP, con una latenza inferiore rispetto ad HLS e DASH;
	\item HTTP Live Streaming (HLS) è il protocollo di streaming ad alta latenza più popolare su HTTP per video on demand (video preregistrato) sviluppato da Apple;
	\item Dynamic Adaptive Streaming over HTTP (DASH) è una tecnica di streaming con bit-rate adattivo del Moving Picture Experts Group (MPEG), che consente lo streaming di alta qualità di contenuti multimediali su protocollo HTTP;
	\item Web Real-Time Communication (WebRTC) è un progetto per la comunicazione in tempo reale basato sul protocollo RTP (Real-time Transport Protocol)\cite{High_Performance_Browser_Networking}.
\end{itemize}

\begin{figure}[H]
	\includegraphics[width=\linewidth]{immagini/webprotocols}
	\caption{API, protocolli e servizi di rete del browser di alto livello}
	\label{fig:webprotocols}
\end{figure}

Il sistema è stato progettato con un'ottica incentrata sull'utilizzo in stand di retro-gaming ad eventi di informatica e videogiochi, da utilizzare quindi sulla rete locale dell'evento con gli utenti connessi tramite WiFi. In questo contesto la differenza di velocità tra TCP e RTP può essere trascurata, e la scelta della tecnologia di streaming da usare è ricaduta su WebSocket perché è un protocollo di comunicazione standardizzato dal 2011, è pienamente supportato da tutti i browser moderni, ha una latenza inferiore rispetto ad HLS e DASH, è semplice da instanziare e non richiede l'utilizzo di protocolli aggiuntivi o configurazioni complesse a differenza di WebRTC.

\begin{figure}[H]
	\includegraphics[width=\linewidth]{immagini/proposed_system}
	\caption{Panoramica del sistema}
	\label{fig:proposed_system}
\end{figure}

Come mostrato in Fig. \ref{fig:proposed_system} il sistema è costituito dal server di gioco, che può essere Linux, Windows o macOS, su cui è in esecuzione la versione modificata di MAME ed una pagina HTML5 che funge da front-end. Il programma è in ascolto per connessioni WebSocket con parametri (per es.: il nome del gioco, l'ID del player, l'ID della partita). Una volta stabilita la connessione, il server invia informazioni sulla dimensione e le proporzioni del video e avvia il gioco. Il rendering e il missaggio audio del gioco vengono generati utilizzando la libreria SDL\footnote{SDL: Simple DirectMedia Layer, una libreria multipiattaforma ed open-soure per il multimedia}, codificati e pacchettizzati nel contenitore MPEG-TS\footnote{MPEG-TS: MPEG transport stream, è un contenitore digitale per la trasmissione e l'archiviazione audio-video.} usando la libreria FFmpeg\footnote{FFmpeg è una suite open-source di librerie e programmi per la gestione di video, audio, e altri file multimediali e stream.} e inviati tramite WebSocket al client.

Lato client vari script si occupano di decodificare i dati audio-video ricevuti, catturare e inviare l'input dell'utente, sia dalla tastiera che dal gamepad, al server tramite WebSocket.



\section{MAME}
Multiple Arcade Machine Emulator (MAME) è un progetto open-source (GNU-GPL) del programmatore italiano Nicola Salmoria. La prima versione del MAME risale al febbraio 1997 ed è attualmente supportato da una vasta comunità di sviluppatori da tutto il mondo. Lo scopo principale è quello di essere un riferimento al funzionamento interno delle macchine emulate, sia per scopi educativi che per scopi di conservazione, al fine di evitare che il software storico scompaia per motivi di obsolescenza. Il progetto è stato realizzato in C++ inizialmente usando principalmente la libreria standard e poi successivamente, negli anni, sono state aggiunte al progetto varie librerie open-source per estenderne le funzionalità. Originariamente era disponibile solo per MS-DOS, ma con il passare del tempo e l'ingrandirsi della comunità di sviluppatori è stato portato per i sistemi Windows e Unix-like\cite{MAME}.

Per quanto riguarda la portabilità ci sono solo 3 compilazioni native differenti e sono quella per macOS, Windows ed UWP\footnote{UWP - Universal Windows Platform è un'architettura applicativa della Microsoft per sviluppare applicazioni eseguibili su Windows 10, Xbox One e Hololens.}, c'è anche una speciale compilazione che deriva da un port del MAME il progetto SDLMAME in grado di funzionare su Windows, Linux, macOS ed altri sistemi operativi supportati dalla libreria SDL.

MAME è un software vastissimo, basti pensare che alla versione attuale (v. 0.228) è formato da 14.904 file sorgenti (e nel conteggio non ci sono i file di definizione della UI, gli script per la compilazione e le librerie esterne che sono incluse nel progetto sottoforma di sorgenti). Il progetto è suddiviso in tre macro categorie, come mostrato in Fig. \ref{fig:mame_arch}, che sono:
\begin{itemize}
	\item Core in cui ci sono i sotto-progetti indipendenti dal sistema e dal device emulato, ed i sorgenti delle librerie esterne;
	\item OSD contenente le funzionalità sia dipendenti dal sistema operativo come macOS, Windows, UWP ed SDLMAME, sia i moduli di input, audio e video dipendenti da altre librerie come OpenGL, DirectX, SDL, CoreAudio, XAudio, XInput, ecc\dots;
	\item Devices che contiene per ogni device emulato le classi che gestiscono le informazioni della macchina ed emulano cpu, bus, video e audio.
\end{itemize} 

\begin{figure}[H]
	\includegraphics[width=\linewidth]{immagini/mame_arch}
	\caption{MAME, diagramma dei packages}
	\label{fig:mame_arch}
\end{figure}


In Fig. \ref{fig:mame_video_class_diagram} è visibile un diagramma delle classi relativo solo alla funzionalità di rendering (la controparte del missaggio audio funziona in maniera simile).

Lorem ipsum dolor sit amet, consectetur adipiscing elit, sed do eiusmod tempor incididunt ut labore et dolore magna aliqua. Ut enim ad minim veniam, quis nostrud exercitation ullamco laboris nisi ut aliquip ex ea commodo consequat. Duis aute irure dolor in reprehenderit in voluptate velit esse cillum dolore eu fugiat nulla pariatur. Excepteur sint occaecat cupidatat non proident, sunt in culpa qui officia deserunt mollit anim id est laborum.

\begin{figure}[H]
	\includegraphics[width=\linewidth]{immagini/mame_video_class_diagram}
	\caption{MAME, diagramma delle classi relativo al rendering}
	\label{fig:mame_video_class_diagram}
\end{figure}

\subsection{SDL}
Simple DirectMedia Layer (SDL) è una libreria multipiattaforma che fornisce accesso di basso livello ad audio, tastiera, mouse, gamepad, hardware 3D e framebuffer 2D. Come mostrato in Fig. \ref{fig:sdl} SDL è costruito sopra le API di visualizzazione video del sistema operativo (in arancione), librerie di rendering 3D (in verde) e librerie che si interfacciano alla scheda audio (in rosso)\cite{SDL_Wiki}.

\begin{figure}[H]
	\includegraphics[width=\linewidth]{immagini/sdl}
	\caption{SDL: livelli di astrazione su diverse piattaforme}
	\label{fig:sdl}
\end{figure}

\subsection{Video}
Il MAME è in grado di emulare giochi sia 2D che 3D (es.: Tekken della Namco), ma poiché emula fisicamente il monitor del cabinato ciò che viene inviato alla libreria grafica è un insieme di primitive e texture da disegnare.

\begin{figure}[H]
	\includegraphics[width=\linewidth]{immagini/rendering_pipeline}
	\caption{Pipeline di rendering 2D}
	\label{fig:rendering_pipeline}
\end{figure}

\subsubsection{SDL renderer} \label{SDL_renderer}
Quando la finestra di gioco viene inizializzata, viene creato un contesto di rendering SDL per la finestra tramite la funzione \textit{CreateRenderer}. Per ogni frame della macchina che viene emulato c'è una fase di disegno usando \textit{SetRenderDrawColor}, \textit{RenderFillRect} e \textit{RenderDrawLine}, e poi tramite la funzione \textit{RenderPresent} viene mostrato il frame appena renderizzato sulla finestra.

\subsection{Audio}
Lorem ipsum dolor sit amet, consectetur adipiscing elit, sed do eiusmod tempor incididunt ut labore et dolore magna aliqua. Ut enim ad minim veniam, quis nostrud exercitation ullamco laboris nisi ut aliquip ex ea commodo consequat. Duis aute irure dolor in reprehenderit in voluptate velit esse cillum dolore eu fugiat nulla pariatur. Excepteur sint occaecat cupidatat non proident, sunt in culpa qui officia deserunt mollit anim id est laborum.

\subsubsection{SDL audio mixer}
Lorem ipsum dolor sit amet, consectetur adipiscing elit, sed do eiusmod tempor incididunt ut labore et dolore magna aliqua. Ut enim ad minim veniam, quis nostrud exercitation ullamco laboris nisi ut aliquip ex ea commodo consequat. Duis aute irure dolor in reprehenderit in voluptate velit esse cillum dolore eu fugiat nulla pariatur. Excepteur sint occaecat cupidatat non proident, sunt in culpa qui officia deserunt mollit anim id est laborum.

\cite{CPP_Primer}
\cite{Computer_Networking_and_the_Internet}
\cite{Ingegneria_del_software}
\cite{Understanding_the_Linux_Kernel}
\cite{Windows_Server_2012}