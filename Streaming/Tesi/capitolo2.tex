%license:BSD-3-Clause
%copyright-holders:Michele Maione
%============================================================
%
%	Piattaforma di cloud gaming per giochi arcade
%
%============================================================
\chapter{Tecnologie}

In questo capitolo 

\section{Sistema proposto}
Una caratteristica fondamentale del sistema è l'usabilità e per questo, lato client, il browser web è stata la scelta più ovvia.
Esistono alcune tecnologie di streaming per i browser web: WebSocket, HLS, DASH e WebRTC\cite{Audio_and_video_delivery}.

WebSocket è un protocollo di comunicazione che fornisce canali di comunicazione full-duplex su una singola connessione TCP, con una latenza inferiore rispetto a HLS e DASH.

HTTP Live Streaming (HLS) è il protocollo di streaming ad alta latenza più popolare su HTTP per video on demand (video preregistrato) sviluppato da Apple.

Dynamic Adaptive Streaming over HTTP (DASH) è una tecnica di streaming bit rate adattiva di Moving Picture Experts Group (MPEG), che consente lo streaming di alta qualità di contenuti multimediali su HTTP.

Web Real-Time Communication (WebRTC) è un progetto per la comunicazione in tempo reale basato sul protocollo SRTP (Secure Real-time Transport Protocol). Pertanto, per ottenere il flusso SRTP nel browser è necessario un server WebRTC\cite{High_Performance_Browser_Networking}.

\begin{figure}[H]
	\includegraphics[width=\linewidth]{immagini/webprotocols}
	\caption{API, protocolli e servizi di rete del browser di alto livello}
	\label{fig:webprotocols}
\end{figure}

Questo progetto è pensato per essere utilizzato in stand di retro-gaming in eventi commerciali per l'industria IT e dei videogiochi, verrà eseguito su una rete locale e gli utenti sono connessi tramite WiFi, quindi la differenza di velocità tra TCP e UDP diventa trascurabile, quindi il la scelta è ricaduta su WebSocket perché è un protocollo di comunicazione standardizzato dal 2011, è pienamente supportato da tutti i browser moderni, è semplice e non richiede l'utilizzo di protocolli aggiuntivi o configurazioni complesse a differenza di WebRTC.

\begin{figure}[H]
	\includegraphics[width=\linewidth]{immagini/proposed_system}
	\caption{Panoramica del sistema}
	\label{fig:proposed_system}
\end{figure}

Il sistema è costituito da un server di gioco, che può essere Linux, Windows o macOS, che esegue MAME e una pagina HTML5 che funge da front-end.

Il programma sta ascoltando una connessione WebSocket che include parametri, come il nome del gioco. Una volta stabilita la connessione, il server invia informazioni sulla dimensione e le proporzioni del video e avvia il gioco. Il rendering e il missaggio audio del gioco vengono generati tramite SDL\footnote{SDL: Simple DirectMedia Layer, una libreria di sviluppo software multipiattaforma progettata per fornire un livello di astrazione hardware per componenti hardware multimediali del computer.}, Codificati e pacchettizzati in MPEG-TS\footnote{MPEG-TS: MPEG transport stream, è un formato contenitore digitale standard per la trasmissione e l'archiviazione di audio e video.} tramite FFmpeg\footnote{FFmpeg è una vasta suite open source di librerie e programmi per la gestione di video, audio, e altri file multimediali e stream.}. I pacchetti vengono inviati tramite WebSocket al client.

Lato client vari script si occuperanno di decodificare i dati audio / video ricevuti, catturare e inviare l'input dell'utente, sia dalla tastiera che dal gamepad, al server tramite WebSocket.

\section{MPEG}
Lorem ipsum dolor sit amet, consectetur adipiscing elit, sed do eiusmod tempor incididunt ut labore et dolore magna aliqua. Ut enim ad minim veniam, quis nostrud exercitation ullamco laboris nisi ut aliquip ex ea commodo consequat. Duis aute irure dolor in reprehenderit in voluptate velit esse cillum dolore eu fugiat nulla pariatur. Excepteur sint occaecat cupidatat non proident, sunt in culpa qui officia deserunt mollit anim id est laborum.

\subsection{Compression}
Lorem ipsum dolor sit amet, consectetur adipiscing elit, sed do eiusmod tempor incididunt ut labore et dolore magna aliqua. Ut enim ad minim veniam, quis nostrud exercitation ullamco laboris nisi ut aliquip ex ea commodo consequat. Duis aute irure dolor in reprehenderit in voluptate velit esse cillum dolore eu fugiat nulla pariatur. Excepteur sint occaecat cupidatat non proident, sunt in culpa qui officia deserunt mollit anim id est laborum.

\subsection{Video}
Lorem ipsum dolor sit amet, consectetur adipiscing elit, sed do eiusmod tempor incididunt ut labore et dolore magna aliqua. Ut enim ad minim veniam, quis nostrud exercitation ullamco laboris nisi ut aliquip ex ea commodo consequat. Duis aute irure dolor in reprehenderit in voluptate velit esse cillum dolore eu fugiat nulla pariatur. Excepteur sint occaecat cupidatat non proident, sunt in culpa qui officia deserunt mollit anim id est laborum.

\subsection{Audio}
Lorem ipsum dolor sit amet, consectetur adipiscing elit, sed do eiusmod tempor incididunt ut labore et dolore magna aliqua. Ut enim ad minim veniam, quis nostrud exercitation ullamco laboris nisi ut aliquip ex ea commodo consequat. Duis aute irure dolor in reprehenderit in voluptate velit esse cillum dolore eu fugiat nulla pariatur. Excepteur sint occaecat cupidatat non proident, sunt in culpa qui officia deserunt mollit anim id est laborum.

\subsection{Trasmission}
Lorem ipsum dolor sit amet, consectetur adipiscing elit, sed do eiusmod tempor incididunt ut labore et dolore magna aliqua. Ut enim ad minim veniam, quis nostrud exercitation ullamco laboris nisi ut aliquip ex ea commodo consequat. Duis aute irure dolor in reprehenderit in voluptate velit esse cillum dolore eu fugiat nulla pariatur. Excepteur sint occaecat cupidatat non proident, sunt in culpa qui officia deserunt mollit anim id est laborum.



\section{FFmpeg}
Lorem ipsum dolor sit amet, consectetur adipiscing elit, sed do eiusmod tempor incididunt ut labore et dolore magna aliqua. Ut enim ad minim veniam, quis nostrud exercitation ullamco laboris nisi ut aliquip ex ea commodo consequat. Duis aute irure dolor in reprehenderit in voluptate velit esse cillum dolore eu fugiat nulla pariatur. Excepteur sint occaecat cupidatat non proident, sunt in culpa qui officia deserunt mollit anim id est laborum\cite{FFmpeg_Documentation}.

\subsection{Libs.}
Lorem ipsum dolor sit amet, consectetur adipiscing elit, sed do eiusmod tempor incididunt ut labore et dolore magna aliqua. Ut enim ad minim veniam, quis nostrud exercitation ullamco laboris nisi ut aliquip ex ea commodo consequat. Duis aute irure dolor in reprehenderit in voluptate velit esse cillum dolore eu fugiat nulla pariatur. Excepteur sint occaecat cupidatat non proident, sunt in culpa qui officia deserunt mollit anim id est laborum.



\section{Simple DirectMedia Layer (SDL)}
SDL è una libreria che fornisce accesso di basso livello ad audio, tastiera, mouse, gamepad, hardware 3D e framebuffer 2D su più piattaforme, anche mobili. SDL è costruito sopra le API di visualizzazione video di O.S., una libreria di rendering 3D e una libreria che si interfaccia alla scheda audio\cite{SDL_Wiki}.

\begin{figure}[H]
	\includegraphics[width=\linewidth]{immagini/sdl}
	\caption{Livelli di astrazione di diverse piattaforme SDL}
	\label{fig:sdl}
\end{figure}

\subsection{Video}
Il MAME è in grado di emulare giochi 3D ma poiché si sta emulando un monitor fisico, ciò che viene inviato alle varie librerie grafiche è un insieme di primitive e texture da disegnare, e per questo motivo il disegno viene sempre eseguito in 2D.

\begin{figure}[H]
	\includegraphics[width=\linewidth]{immagini/rendering_pipeline}
	\caption{Pipeline di rendering 2D}
	\label{fig:rendering_pipeline}
\end{figure}

Quando la finestra viene inizializzata, viene creato un contesto di rendering 2D SDL per la finestra tramite la funzione \textit{CreateRenderer}. Per ogni frame di macchina emulato c'è una fase di disegno usando \textit{SetRenderDrawColor}, \textit{RenderFillRect} e \textit{RenderDrawLine}. Alla fine la funzione \textit{RenderPresent} viene utilizzato per mostrare la cornice sulla finestra.

Nella sezione \ref{SDL_renderer} vengono introdotte altre funzioni, usate per modificare questo comportamento standard descritto sopra.


\subsection{Audio}
Lorem ipsum dolor sit amet, consectetur adipiscing elit, sed do eiusmod tempor incididunt ut labore et dolore magna aliqua. Ut enim ad minim veniam, quis nostrud exercitation ullamco laboris nisi ut aliquip ex ea commodo consequat. Duis aute irure dolor in reprehenderit in voluptate velit esse cillum dolore eu fugiat nulla pariatur. Excepteur sint occaecat cupidatat non proident, sunt in culpa qui officia deserunt mollit anim id est laborum.



\section{Web APIs}
Le API Web sono un insieme di API e interfacce che comprendono la potente capacità di creazione di script del Web. A seguire quelli utilizzati in questo progetto\cite{Web_APIs}.

\subsection{WebSocket}
WebSocket è un protocollo di comunicazione del computer che fornisce canali di comunicazione full-duplex su una singola connessione TCP. È compatibile con HTTP perché l'handshake WebSocket utilizza l'intestazione di aggiornamento HTTP per passare dal protocollo HTTP al protocollo WebSocket. È supportato nativamente da tutti i browser e il suo utilizzo è simile ai normali socket sia sul lato client che su quello server. Per questi motivi è il protocollo di comunicazione generico più utilizzato sul web\cite{WebSocket_Web_APIs}.

\subsection{Canvas API}
L'API Canvas fornisce un mezzo per disegnare grafica tramite JavaScript, si concentra principalmente sulla grafica 2D ma quando viene utilizzata dall'API WebGL può disegnare grafica 2D e 3D con accelerazione hardware. È completamente supportato da tutti i browser\cite{Canvas_API}.

\subsection{WebGL API}
WebGL è un'API JavaScript, progettata e gestita dal gruppo no-profit Khronos, per il rendering di grafica 2D e 3D interattiva che consente l'utilizzo accelerato dalla GPU della fisica e dell'elaborazione e degli effetti delle immagini. WebGL 1.0 è supportato su tutti i browser, mentre WebGL 2.0 viene testato su Safari\cite{WebGL}.



\section{JavaScript libraries}
Per il front-end, sono state utilizzate librerie JavaScript open source per la gestione degli input e per la decodifica del filmato.

\subsection{JSMpeg}
JSMpeg è una libreria JavaScript composta da un demuxer MPEG-TS, video MPEG1 e decoder audio MP2, renderer WebGL e Canvas2D e output audio WebAudio. JSMpeg può caricare video statici tramite Ajax e consente lo streaming a bassa latenza ($\sim$50ms) tramite WebSocket, è rilasciato con licenza MIT\cite{JSMpeg}.

\subsection{Keypress}
Keypress è un'utilità JavaScript che cattura l'input da tastiera focalizzata sull'input per i giochi, rilasciata con la licenza Apache 2.0. Viene utilizzato per gestire l'input da tastiera nel front-end\cite{Keypress}.

\subsection{GameController.js}
GameController.js è una libreria che utilizza JavaScript e l'API Gamepad standard, è rilasciata con licenza MIT. Nel front-end viene utilizzato per gestire i gamepad, per consentire il multiplayer da divano\cite{gameController_js}.